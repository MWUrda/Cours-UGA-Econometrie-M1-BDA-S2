\input{../../Preambule}
\newcommand{\bxi}{ \boldx_i }
\usepackage{tikz}
\usepackage{enumitem}


\usepackage{fancyhdr}
\pagestyle{fancy}
%\renewcommand{\subsection{mark}[1]{\markright{#1}{}}
\fancyhead{}
\fancyfoot{} 
%\fancyhead[LE,LO]{\tiny{\thepage}}
\fancyhead[C]{\small\textsc{Économétrie}}
%fancyhead[CE,CO]{\tiny{\rightmark}}
\fancyhead[L]{\small\textsc{UGA}}
\fancyfoot[C]{\small{\thepage}}
%\fancyfoot[R]{\small \textcopyright \ \  \small\textsc
\fancyhead[R]{ \small\textsc{M. Urdanivia}}
%\renewcommand{\headrulewidth}{0pt}
\renewcommand{\footrulewidth}{0pt}

%\pagenumbering{roman}


\begin{document} 
\usetikzlibrary{positioning}
\usetikzlibrary{snakes}
\usetikzlibrary{calc}
\usetikzlibrary{arrows}
\usetikzlibrary{decorations.markings}
\usetikzlibrary{shapes.misc}
\usetikzlibrary{shapes}
%\tikzset{block/.style={draw, rectangle, fill=gray!20, 
%\tikzset{empty/.style={draw, rectangle, fill=none, tex
%\tikzset{line/.style={draw, -latex'}}
%\onehalfspace

\begin{titlepage}
\centering
	%\includegraphics[width=0.15\textwidth]{logoUGA2020
	{\scshape\Large \textsc{Économétrie: UGA}\par}
	\vspace{1cm}
	%{\scshape\large \textsc{Extremum Estimators(1)}\par}
	%\vspace{1cm}
	{\Large\bfseries \textsc{4: SYSTÈMES D’ÉQUATIONS SIMULTANÉES} \par}
    \vspace{1cm}   
    {\Large\bfseries \textsc{Devoir 1} \par}
	\vspace{1cm}
	{(\textsc{Cette version: \today})\par}
	\vspace{1cm}
	{\large \textsc{Michal Urdanivia}
	\footnote{Contact:  
	\href{mailto:michal.wong-urdanivia@univ-grenoble-alpes.fr}{michal.wong-urdanivia@univ-grenoble-alpes.fr}, 
	 Université de Grenoble Alpes,  Faculté d'\'Economie, GAEL.}\par}
	%\vfill
	%supervised by\par
	%Dr.~Mark \textsc{Brown}
%\vfill
% Bottom of the page
	
\end{titlepage}


\newpage

\tableofcontents

\newpage

\section{Objectifs}
\begin{enumerate}
\item Illustrer les mécanismes d’identification dans les systèmes d’équations simultanées et leurs liens avec les techniques de VI
\item Manipuler des notations matriciels
\item Montrer comment utiliser un résultat théorique en pratique, ici la méthode du delta pour déterminer la distribution as. 
d’un estimateur des MCI.
\item Appliquer sur des données certaines des méthodes.
\end{enumerate}

\section{Théorie}

\subsection{Modèles récursifs}
Les vecteurs de variables aléatoires $(y_{1i},y_{2i},z_{1i},z_{2i})$ sont indépendants, identiquement distribués pour 
$i= 1,\ldots, N$. Ils satisfont en outre les conditions de régularité requises par les applications de la loi
 des grands nombres et du théorème central limite nécessaires pour la normalité as. des estimateurs considérés 
 (lorsqu’ils sont employés de manière pertinente!). 

 On considère ici différentes version du modèle suivant:

    
 \begin{align}
    \left\{
 \begin{array}{l}
    y_{1i} = a_{10}z_{1i} + a_{2,0}z_{2i} + a_{3,0}y_{2i} +u_{1i}\\
    y_{2i}= b_{1,0}z_{1i} + b_{2,0}z_{2i}+u_{2i}
 \end{array}
\right.
&, \ \text{avec} \  \Exp[\boldu_i|\boldz_i] =0,  \text{et} \ \Exp[\boldu_i\boldu_i^\prime|\boldz_i]=\boldOmega
\label{eq1}
 \end{align}

 avec les notations suivantes:

 \[
    \boldy_i \equiv 
    \begin{bmatrix}
        y_{1i}\\
        y_{2i}
    \end{bmatrix}
    , \quad 
    \boldz_i \equiv 
\begin{bmatrix}
    z_{1i}\\
    z_{2i}
\end{bmatrix}
, \quad 
\boldu_i \equiv 
\begin{bmatrix}
    u_{1i}\\
    u_{2i}
\end{bmatrix}
, \quad 
\boldOmega \equiv
\begin{bmatrix}
    \omega_{11}&\omega_{12}\\
    \omega_{12}&\omega_{22}
\end{bmatrix}
 \]
    
 On utilisera également les notations:


 \begin{align*}
    \bolda_0 \equiv 
    \left[
    \begin{array}{l}
        a_{1,0}\\
        a_{2, 0}\\
        a_{3, 0}
    \end{array}
    \right]
    , \quad 
    \boldb_0 \equiv 
 \left[
 \begin{array}{l}
    b_{1,0}\\
    b_{2,0}
\end{array}
 \right]
\end{align*}

et:

\[
    \underline{\boldy}_k \equiv 
    \begin{bmatrix}
        y_{k,1}\\
        \vdots\\
        y_{k, N}
    \end{bmatrix}_{N\times 1}
    , \quad 
    \underline{\boldz}_k \equiv
        \begin{bmatrix}
        z_{k,1}\\
        \vdots\\
        z_{k, N}
    \end{bmatrix}_{N\times 1}
    , \quad \text{pour $k=1, 2$ et} \
    \boldZ \equiv 
\begin{bmatrix}
    \underline{\boldz}_1& \underline{\boldz}_2
\end{bmatrix}_{N\times 2},
\]
et on suppose que $\Var[\boldz_i]$ est inversible.

\begin{enumerate}
    \item Justifier l’absence de paramètres constants pour les modèles de $y_{1i}$ et $y_{2i}$ 
    \item  Déterminer les variables exogènes et endogènes de chacune des équations
    du système et du système
    \item Qualifier le modèle décrit par le système d’équations \eqref{eq1}.
    \item Analyser l’identification de $\boldb_0$ et $\bolda_0$ dans le modèle.  
    \item Proposer un estimateur convergent de $\boldb_0$ et donner sa forme avec les $y_{2,i}$ et $\boldz_i$ 
    ainsi qu’avec $\underline{\boldy}_k$ et $\boldZ$.
    \item Donner une approximation de la distribution de $\boldb_0$ lorsque $N$ est grand et un estimateur de sa précision.
    \item On suppose que $\omega_{12} = 0$ dans le modèle \eqref{eq1}:
    \begin{enumerate}
        \item Commentez cette condition. 
        \item Montrer qu’on a alors $\Cov[y_{2i}; u_{1i}]= 0$.
        \item Quelles sont les implications de $\Cov[y_{2i}; u_{1i}]= 0$ pour $y_{2i}$ et pour l’estimation de $\bolda_0$?
        \item Justifiez que $\omega_{12} = 0$ soit une condition peut vraisemblable dans le modèle.
    \end{enumerate}
    \item On suppose que dans le modèle \eqref{eq1} $a_{1, 0} = 0$ de sorte que l'on étudie:
    \begin{align}
        \left\{
     \begin{array}{ll}
        y_{1i} =  a_{2,0}z_{2i} + a_{3,0}y_{2i} +u_{1i}\\
        y_{2i}= b_{1,0}z_{1i} + b_{2,0}z_{2i}+u_{2i}
     \end{array}
    \right.
    &, \ \text{avec} \  \Exp[\boldu_i|\boldz_i] =0,  \text{et} \ \Exp[\boldu_i\boldu_i^\prime|\boldz_i]=\boldOmega
    \label{eq2}
     \end{align}
     et on note:
     \[
        \bolda_0 \equiv 
        \begin{bmatrix}
            a_{2, 0}\\
            a_{3, 0}
        \end{bmatrix}
    \]
\begin{enumerate}
    \item Qu’implique $a_{1,0}=0$ pour pour $z_{1i}$?
    \item Qu’implique $a_{1,0}=0$ pour $y_{2i}$ dans l’équation de $y_{1i}$?
    \item Montrer que $\boldb_0$ et $\bolda_0$ sont juste-identifiés dans le modèle \eqref{eq2} (à certaines conditions) ?
    \item Proposer deux méthodes d’estimation de $\boldb_0$ et $\bolda_0$ et définir les estimateurs
    (convergents) correspondants.
    On utilisera les notations:
    \[
        \boldx_{1, i}\equiv \begin{bmatrix}
            \boldz_{2i}, y_{2i}
        \end{bmatrix}
        \ \text{et} \
        \boldX_1 \equiv  \begin{bmatrix}
            \underline{\boldz}_2&\underline{\boldy}_2
        \end{bmatrix}_{N\times 2}
        \]
\end{enumerate}
\item On suppose que pour le modèle \eqref{eq1} $a_{1,0}= a_{2,0}= b_{2,0}=  0$. Montrez qu'alors 
les estimateurs des VI et des MCI de $a_{3,0}$ et $b_{1,0}$ sont égaux.
\end{enumerate}
\newpage
\subsection{Modèles non récursifs}
Les vecteurs de variables aléatoires $(y_{1i} , y_{2i} , z_{1i} , z_{2i} , z_{3i})$ sont indépendants, identiquement
distribués pour $i=1,\ldots, N$. Ils satisfont en outre les conditions de régularité requises 
par les applications de la loi des grands nombres et du théorème central limite nécessaires 
pour la normalité as. des estimateurs considérés (lorsqu’ils sont employés de manière pertinente).
On considère ici différentes version du modèle suivant:
\begin{align}
    \left\{
        \begin{array}{l}
    y_{1i} =  a_{1,0}z_{1i}+ a_{2,0}z_{2i} + a_{3,0}z_{3i} + a_{4,0}y_{2i} + u_{1i}\\
    y_{2i} = b_{1,0}z_{1i} + b_{2,0}z_{2i} + b_{3,0}z_{3i} + b_{4,0}y_{1i} + u_{2i}
        \end{array}
        \right. , 
        \ \text{avec} \ \Exp[\boldu_i|\boldz_i] = 0 \ \text{et} \ \Exp[\boldu_i\boldu_i^\prime|\boldz_i]=\boldOmega
        \label{eq3}
\end{align}
    On notera:
\[
    \boldy_i \equiv \begin{bmatrix}
        y_{1i}\\
        y_{2i}
    \end{bmatrix}
    , \
    \boldz_i \equiv \begin{bmatrix}
        z_{1i}\\
        z_{2i}\\
        z_{3i}
    \end{bmatrix}
    , \
    \boldu_i \equiv \begin{bmatrix}
        u_{1i}\\
        u_{2i}
    \end{bmatrix}
    , \ 
    \boldOmega \equiv \begin{bmatrix}
        \omega_{11} & \omega_{12}\\
        \omega_{12} & \omega_{22}
    \end{bmatrix}
    \]
    et aussi:
    \[
    \bolda_0 \equiv \begin{bmatrix}
        a_{1, 0}\\
        a_{2, 0}\\
        a_{3, 0}\\
        a_{4, 0}
        \end{bmatrix}    
        , \  
        \boldb_0 \equiv \begin{bmatrix}
            b_{1, 0}\\
            b_{2, 0}\\
            b_{3, 0}\\
            b_{4, 0}
            \end{bmatrix}   
            , \ 
            \boldx_{1i} \equiv \begin{bmatrix}
                \boldz_i\\
                y_{2i}
                \end{bmatrix}     
                , \ 
                \boldx_{2i} \equiv \begin{bmatrix}
                    \boldz_i\\
                    y_{1i}
                    \end{bmatrix}     
    \],
    
    et supposera que $\Var[\boldz_i]$ est inversible.
\begin{enumerate}
    \item Déterminer les variables exogènes et endogènes de chacune des équations
    du système et du système.
    \item Qualifier le modèle décrit par le système d’équations \eqref{eq3}.
    \item Définir $\boldGamma_0$ et $\boldG_0$ pour écrire le modèle \eqref{eq3} sous la forme matrielle matricielle:
    \begin{align*}
    \boldGamma_0\boldy_i&= \boldG_0\boldz_i + \boldu_i.
    \end{align*}
    \item Déterminer la forme réduite du modèle \eqref{eq3} en la notant: 
    \begin{align*}
        \boldy_i &=\boldPi\boldz_i + \boldv_i,
    \end{align*}
où 

\[\boldv_i \equiv \begin{bmatrix}
v_{1i}\\
v_{2i}
\end{bmatrix}
\]
sont les erreur du modèle réduit. On utilisera aussi la notation:
\[  
    \boldPi \equiv \begin{bmatrix}
        \pi_{11}& \pi_{12}& \pi_{13}\\
        \pi_{21}& \pi_{22}& \pi_{23}
    \end{bmatrix}
    \equiv 
    \begin{bmatrix}
\boldpi_1^\prime\\
\boldpi_2^\prime
    \end{bmatrix}
\]
\item Analyser l’identification des paramètres $\boldb_0$ et $\bolda_0$.
\item On suppose que dans le modèle \eqref{eq3} $a_{1,0}= a_{2,0}= a_{3,0}= 0$.
\begin{enumerate}
\item Montrer que $a_{4,0}$ est le seul élément identifiable de $\bolda_0$ et $\boldb_0$.  
\item Proposer un estimateur convergent de $a_{4,0}$ et de $\omega_{11}$. On utilisera les notations:
\[
  \underline{\boldy}_k \equiv
  \begin{bmatrix}
      y_{k, 1}\\
      \vdots\\
      y_{k, N}
  \end{bmatrix}_{N\times 1} 
  \ \text{pour} \ k = 1, 2.
\]

et:

\[
    \underline{\boldz}_k \equiv
    \begin{bmatrix}
        z_{k, 1}\\
        \vdots\\
        z_{k, N}
    \end{bmatrix}_{N\times 1} 
    \ \text{pour} \ k = 1, 2, 3, \ \text{et aussi} \ 
    \boldZ\equiv \begin{bmatrix}
        \underline{\boldz}_1&\underline{\boldz}_2&\underline{\boldz}_3
    \end{bmatrix}_{N\times 3} .
\]
\item Montrer que $\hat{a}_{4, N}^{2MC}$ utilise $\hat{\boldpi}_{2, N}^{MCO}$, l’estimateur des MCO de $\boldpi_2$ estdans la forme réduite 
du modèle de $y_{2i}$.
\item Montrer que $\hat{a}_{4, N}^{2MC}$ utilise un estimateur de la projection linéaire de $y_{2i}$ sur $\boldz_i$.
\end{enumerate}
\item On suppose maintenant que dans le modèle \eqref{eq3} $a_{1,0}= a_{2,0}=  0$ et $b_{3,0} = 0$.
 Et on utilisera dorénavant les notations:

 \[
 \bolda_0 \equiv \begin{bmatrix}
    a_{3, 0}\\
    a_{4, 0}
 \end{bmatrix}
, \ 
\boldb_0 \equiv \begin{bmatrix}
    b_{1, 0}\\
    b_{2, 0}\\
    b_{4, 0}
 \end{bmatrix}
 , \ 
 \boldx_{1i} \equiv \begin{bmatrix}
    z_{3i}\\
    y_{2i}
 \end{bmatrix}
 , \ 
 \boldx_{2i} \equiv \begin{bmatrix}
    z_{1i}\\
    z_{2i}\\
    y_{1i}
 \end{bmatrix}.
 \]
 \begin{enumerate}
\item Montrer que $\bolda_0 \equiv (a_{3,0}, a_{4,0})$ et $(\boldb_0 \equiv(b_{1,0},b_{2,0}, b_{4,0})$ sont 
identifiables à priori.
\item Proposer des estimateurs convergents et rapidement calculables de $\bolda_0$ et $\boldb_0$.
\item Montrer que l’estimateur des 2MC de $\boldb_0$ dans l’équation de $y_{2i}$ 
avec $\boldz_i$ pour vecteur d’instruments est égal à l’estimateur des VI correspondant. 
\item Proposer un estimateur convergent de $\boldOmega$ et écrire la forme de cet estimateur.
 \end{enumerate}
 \item On suppose maintenant que dans \eqref{eq3} $a_{1,0} = a_{2,0}= a_{3,0}= b_{2,0}= b_{3,0}= b_{4,0}=0$.
\begin{enumerate}
\item Déterminer la forme réduite du modèle \eqref{eq3}) sous cette condition sur les paramètres. Cela sous la forme: 

\begin{align*}
 \boldy_i &= \boldpi z_{1i} + \boldv_i = \begin{bmatrix}
     \pi_1\\
     \pi_2
 \end{bmatrix}
 z_{1i}+ \boldv_i
\end{align*}
\item Déterminer les paramètres de la forme structurelle en fonction de ceux de la forme réduite 
et en déduire l’estimateur des MCI de $a_{4, 0}$ noté $\hat{a}_{4, N}^{MCI}$.
\item Déterminer la distribution as. du vecteur:
\[\hat{\boldpi}_N \equiv
    \begin{bmatrix}
        \hat{\pi}_1^{MCO}\\
        \hat{\pi}_2^{MCO}
    \end{bmatrix}
    \]
    \item Utiliser la propriété suivante, dite "Méthode du delta", pour déterminer la distribution as. de
    $\hat{a}_{4, N}^{MCI}$.

    \begin{remarque}(\textbf{\underline{Rappel sur Méthode du Delta}})
        Si:
        \begin{align*}
            \sqrt{N}\left(\bolda_N - \bolda_0\right)&\limloi \mathcal{N}\left(\boldzero, \boldSigma_0\right).
        \end{align*}
        et si $\boldg(\bolda)$ est continûment différentiable sur le domaine de définition de $\bolda$ alors:
\begin{align*}
    \sqrt{N}\left(\boldg(\bolda_N)-\boldg(\bolda_0)\right)&\limloi
     \mathcal{N}\left(\boldzero, 
    \frac{\partial \bold g}{\partial \bolda^\prime}(\bolda_0)\boldSigma_0 \frac{\partial \bold g}{\partial \bolda}(\bolda_0)^\prime\right).
\end{align*}
    \end{remarque}
\end{enumerate}
\end{enumerate}
\bibliographystyle{jpe}
\bibliography{../../Biblio.bib}
 \end{document}
