%\documentclass[ignorenonframetext, compress, 9pt, xcolor=svgnames]{beamer} 
\documentclass[notes, ignorenonframetext, compress, 11pt, xcolor=svgnames, aspectratio=169]{beamer} 
\usepackage{pgfpages}
\usepackage{pdfpages}
% These slides also contain speaker notes. You can print just the slides,
% just the notes, or both, depending on the setting below. Comment out the want
% you want.
\setbeameroption{hide notes} % Only slide
%\setbeameroption{show only notes} % Only notes
%\setbeameroption{show notes on second screen=right} % Both
\usepackage{amsmath}
\usepackage{amsfonts}
\usepackage{amssymb}
\setbeamercolor{frametitle}{fg=MidnightBlue}

\setbeamercolor{sectionpage title}{bg=MidnightBlue}
\setbeamertemplate{frametitle}[default][center]
%\setbeamertemplate{frametitle}{\color{MidnightBlue}\centering\bfseries\insertframetitle\par\vskip-6pt}
\setbeamerfont{frametitle}{series=\bfseries}
\setbeamerfont{title}{series=\bfseries}
\setbeamerfont{sectionpage}{series=\bfseries}
%\setbeamercolor{section in head/foot}{bg=MidnightBlueBlue}
%\setbeamercolor{author in head/foot}{bg=DarkBlue}
\setbeamercolor{author in head/foot}{fg=MidnightBlue}
%\setbeamercolor{title in head/foot}{bg=White}
\setbeamercolor{title in head/foot}{fg=MidnightBlue}
\setbeamercolor{title}{fg=MidnightBlue}
%\setbeamercolor{date in head/foot}{fg=Brown}
%\setbeamercolor{alerted text}{fg=DarkBlue}
%\usecolortheme[named=DarkBlue]{structure} 
%\usepackage{bbm}
%\usepackage{bbold}
\usepackage{eurosym}
\usepackage{graphicx}
%\usepackage{epstopdf}
\usepackage{hyperref}
\hypersetup{
  colorlinks   = true, %Colours links instead of ugly boxes
  urlcolor     = gray, %Colour for external hyperlinks
  linkcolor    = MidnightBlue, %Colour of internal links
  citecolor   = DarkRed %Colour of citations
}
\usepackage{multirow}
\usepackage{xspace}
\usepackage{listings}
\usepackage{natbib}
%\usepackage[sort&compress,comma,super]{natbib}
\def\newblock{} % To avoid a compilation error about a function \newblock undefined
\usepackage{bibentry}
\usepackage{booktabs}
\usepackage{dcolumn}
\usepackage[greek,frenchb]{babel}
\usepackage[babel=true,kerning=true]{microtype}
\usepackage[utf8]{inputenc}
\usepackage[T1]{fontenc}
\usepackage{natbib}
\renewcommand{\cite}{\citet}
\usepackage{longtable}
\usepackage{eso-pic}

\usepackage{xcolor}
 \colorlet{linkequation}{DarkRed} 
 \newcommand*{\SavedEqref}{}
 \let\SavedEqref\eqref 
\renewcommand*{\eqref}[1]{%
\begingroup \hypersetup{
      linkcolor=linkequation,
linkbordercolor=linkequation, }%
\SavedEqref{#1}%
 \endgroup
}

\newcommand*{\refeq}[1]{%
 \begingroup
\hypersetup{ 
linkcolor=linkequation, 
linkbordercolor=linkequation,
}%
\ref{#1}%
 \endgroup
}

\setbeamertemplate{caption}[numbered]
\setbeamertemplate{theorem}[ams style]
\setbeamertemplate{theorems}[numbered]
%\usefonttheme{serif}
%\usecolortheme{beaver}
%\usetheme{Hannover}
%\usetheme{CambridgeUS}
%\usetheme{Madrid}
%\usecolortheme{whale}
%\usetheme{Warsaw}
%\usetheme{Luebeck}
%\usetheme{Montpellier}
%\usetheme{Berlin}
%\setbeamercolor{titlelike}{parent=structure}
%\setbeamertemplate{headline}[default]
%\setbeamertemplate{footline}[default]
%\setbeamertemplate{footline}[Malmoe]
%\setbeamercovered{transparent}
%\setbeamercovered{invisible}
%\usecolortheme{crane}
%\usecolortheme{dolphin}
%\usepackage{pxfonts}
%\usepackage{isomath}
%\usepackage{mathpazo}
%\usepackage{arev} %     (Arev/Vera Sans)
%\usepackage{eulervm} %_   (Euler Math)
%\usepackage{fixmath} %  (Computer Modern)
%\usepackage{hvmath} %_   (HV-Math/Helvetica)
%\usepackage{tmmath} %_   (TM-Math/Times)
%\usepackage{tgheros}
%\usepackage{cmbright}
%\usepackage{ccfonts} \usepackage[T1]{fontenc}
%\usepackage[garamond]{mathdesign}

%\usepackage{color}
%\usepackage{ulem}

%\usepackage[math]{kurier}
%\usepackage[no-math]{fontspec}
%\setmainfont{Fontin Sans}
%\setsansfont{Fontin Sans}
%\setbeamerfont{frametitle}{size=\LARGE,series=\bfseries}
%%%add 19022021
\usepackage{enumerate}    
\usepackage{dcolumn}
\usepackage{verbatim}
\newcolumntype{d}[0]{D{.}{.}{5}}
%\setbeamertemplate{note page}{\pagecolor{yellow!5}\insertnote}
%\usetikzlibrary{positioning}
%\usetikzlibrary{snakes}
%\usetikzlibrary{calc}
%\usetikzlibrary{arrows}
%\usetikzlibrary{decorations.markings}
%\usetikzlibrary{shapes.misc}
%\usetikzlibrary{matrix,shapes,arrows,fit,tikzmark}
%%%
% suppress navigation bar
\beamertemplatenavigationsymbolsempty
%\usetheme{bunsenMod}
%\setbeamercovered{transparent}
%\setbeamertemplate{items}[circle]
%\usecolortheme[named=CadetBlue]{structure}
%\usecolortheme[RGB={225,64,5}]{structure}
%\definecolor{burntRed}{RGB}{225,64,5}
%\setbeamercolor{alerted text}{fg=burntRed} 
%\usecolortheme[RGB={0,40,110}]{structure}
%\hypersetup{linkcolor=burntRed}
%\hypersetup{urlcolor=burntRed}
%\hypersetup{filecolor=burntRed}
%\hypersetup{citecolor=burntRed}

%\usetheme{bunsenMod}
%\setbeamercovered{transparent}
%\setbeamertemplate{items}[circle]
%\usecolortheme[named=CadetBlue]{structure}
%\usecolortheme[RGB={225,64,5}]{structure}
%\definecolor{burntRed}{RGB}{225,64,5}
%\setbeamercolor{alerted text}{fg=burntRed} 
%\usecolortheme[RGB={0,40,110}]{structure}
%\hypersetup{linkcolor=burntRed}
%\hypersetup{urlcolor=burntRed}
%\hypersetup{filecolor=burntRed}
%\hypersetup{citecolor=burntRed}

%\AtBeginSection[] % Do nothing for \section*
%{ \frame{\sectionpage} }
%\setbeamertemplate{frametitle continuation}{}
\newtheorem{lemme}{Lemme}[section]
%\newtheorem{remarque}{Remarque}
\newcommand{\argmax}{\operatornamewithlimits{arg\,max}}
\newcommand{\argmin}{\operatornamewithlimits{arg\,min}}
\def\inprobLOW{\rightarrow_p}
\def\inprobHIGH{\,{\buildrel p \over \rightarrow}\,} 
\def\inprob{\,{\inprobHIGH}\,} 
\def\indist{\,{\buildrel d \over \rightarrow}\,} 
\def\sima{\,{\buildrel a \over \sim}\,} 
\def\F{\mathbb{F}}
\def\R{\mathbb{R}}
\def\N{\mathbb{N}}
\newcommand{\gmatrix}[1]{\begin{pmatrix} {#1}_{11} & \cdots &
    {#1}_{1n} \\ \vdots & \ddots & \vdots \\ {#1}_{m1} & \cdots &
    {#1}_{mn} \end{pmatrix}}
\newcommand{\iprod}[2]{\left\langle {#1} , {#2} \right\rangle}
\newcommand{\norm}[1]{\left\Vert {#1} \right\Vert}
\newcommand{\abs}[1]{\left\vert {#1} \right\vert}
\renewcommand{\det}{\mathrm{det}}
\newcommand{\rank}{\mathrm{rank}}
\newcommand{\spn}{\mathrm{span}}
\newcommand{\row}{\mathrm{Row}}
\newcommand{\col}{\mathrm{Col}}
\renewcommand{\dim}{\mathrm{dim}}
\newcommand{\prefeq}{\succeq}
\newcommand{\pref}{\succ}
\newcommand{\seq}[1]{\{{#1}_n \}_{n=1}^\infty }
\renewcommand{\to}{{\rightarrow}}
\renewcommand{\L}{{\mathcal{L}}}
\newcommand{\Er}{\mathrm{E}}
\renewcommand{\Pr}{\mathrm{P}}
%\newcommand{\Var}{\mathrm{Var}}
%\newcommand{\Cov}{\mathrm{Cov}}
%\newcommand{\corr}{\mathrm{Corr}}
%\newcommand{\Var}{\mathrm{Var}}
\newcommand{\bias}{\mathrm{Bias}}
\newcommand{\mse}{\mathrm{MSE}}
\providecommand{\Pred}{\mathcal{P}}
\providecommand{\plim}{\operatornamewithlimits{plim}}
\providecommand{\avg}{\frac{1}{n} \underset{i=1}{\overset{n}{\sum}}}
\providecommand{\sumin}{{\sum_{i=1}^n}}
\providecommand{\limp}{\overset{p}{\rightarrow}}
%\providecommand{\limp}{\underset{n \rightarrow \infty}{\overset{p}{\longrightarrow}}}
%\providecommand{\limp}{\underset{n \rightarrow \infty}{\overset{p}{\longrightarrow}}}
%\providecommand{\limp}{\overset{p}{\longrightarrow}}
%\providecommand{\limd}{\underset{n \rightarrow \infty}{\overset{d}{\longrightarrow}}}
\providecommand{\limd}{\overset{d}{\rightarrow}}
\providecommand{\limps}{\overset{p.s.}{\rightarrow}}
\providecommand{\limlp}{\overset{L^p}{\rightarrow}}
\def\independenT#1#2{\mathrel{\setbox0\hbox{$#1#2$}%
    \copy0\kern-\wd0\mkern4mu\box0}} 
\newcommand\indep{\protect\mathpalette{\protect\independenT}{\perp}}


\lstset{language=R}
\lstset{keywordstyle=\color[rgb]{0,0,1},                                        % keywords
        commentstyle=\color[rgb]{0.133,0.545,0.133},    % comments
        stringstyle=\color[rgb]{0.627,0.126,0.941}      % strings
}       
\lstset{
  showstringspaces=false,       % not emphasize spaces in strings 
  columns=fixed,
  numbersep=3mm, numbers=left, numberstyle=\tiny,       % number style
  frame=none,
  framexleftmargin=5mm, xleftmargin=5mm         % tweak margins
}
\makeatletter
%\setbeamertemplate{frametitle continuation}{\gdef\beamer@frametitle{}}
\setbeamertemplate{frametitle continuation}{\frametitle{}}
%\setbeamertemplate{frametitle continuation}{\insertcontinuationcount}
\makeatother

\theoremstyle{remark}
\newtheorem{interpretation}{Interprétation}
\newtheorem*{interpretation*}{Interprétation}

\theoremstyle{remark}
\newtheorem{remarque}{Remarque}%[section]
\newtheorem*{remarque*}{Remarque}
\usepackage[framemethod=TikZ]{mdframed} 
\usepackage{showexpl}
%\newtheorem{step}{Step}[section]
%\newtheorem{rem}{Comment}[section]
%\newtheorem{ex}{Example}[section]
%\newtheorem{hist}{History}[section]
%\newtheorem*{ex*}{Example}
\theoremstyle{plain}
\newtheorem{propriete}{Propri\'et\'e}
\renewcommand{\thepropriete}{P\arabic{propriete}}
%\theoremstyle{definition}
%\newtheorem{definition}{Définition}%[section]
\theoremstyle{remark}
\newtheorem{exemple}{Exemple}
\newtheorem*{exemple*}{Exemple}

\newtheorem{theoreme}{Théorème}
\newtheorem{proposition}{Proposition}
%\newtheorem{propriete}{Propri\'et\'e}
\newtheorem{corollaire}{Corollaire}
%\newtheorem{exemple}{Exemple}
\newtheorem{assumption}{Assumption}
\renewcommand{\theassumption}{A\arabic{assumption}}
\newtheorem{hypothese}{Hypothèse}
\renewcommand{\thehypothese}{H\arabic{hypothese}}
\theoremstyle{definition}

%\newtheorem{definitionx}{D\'efinition}%[section]
%\newenvironment{definition}
 %{\pushQED{\qed}\renewcommand{\qedsymbol}{$\triangle$}\definitionx}
 %{\popQED\enddefinitionx}

\newtheorem{condition}{Condition}
\renewcommand{\thecondition}{C\arabic{condition}}
%\newcommand{\Var}{\mathbb{V}}
%\newcommand{\Var}{\mathbf{Var}}
%\newcommand{\Exp}{\mathbf{E}}
%\providecommand{\Vr}{\mathrm{Var}}
%\renewcommand{\Er}{\mathbb{E}}
%\newcommand{\LP}{\mathcal{LP}}
%\providecommand{\Id}{\mathbf{I}}
%\providecommand{\Rang}{\mathrm{Rang}}
%\providecommand{\Trace}{\mathrm{Trace}}
%\newcommand{\Cov}{\mathbf{Cov}}
%\newcommand{\Cov}{\mathbb{C}\mathrm{ov}}
\providecommand{\Id}{\mathbf{I}}
\providecommand{\Ind}{\mathbf{1}}
\providecommand{\uvec}{\mathbf{1}}
\providecommand{\vecOnes}{\mathbf{1}}
\DeclareMathOperator{\indfun}{\mathbf{1}}
\DeclareMathOperator{\Exp}{E}
\DeclareMathOperator{\Expn}{\mathbb{E}_n}
\DeclareMathOperator{\Var}{Var}
\DeclareMathOperator{\Vr}{V}
\DeclareMathOperator{\Cov}{Cov}
\DeclareMathOperator{\corr}{corr}
\DeclareMathOperator{\perps}{\perp_s}
%\DeclareMathOperator{\Prob}{Pr}
\DeclareMathOperator{\Prob}{P}
\DeclareMathOperator{\prob}{p}
\DeclareMathOperator{\loss}{L}
\providecommand{\Corr}{\mathrm{Corr}}
\providecommand{\Diag}{\mathrm{Diag}}
\providecommand{\reg}{\mathrm{r}}
\providecommand{\Likelihood}{\mathrm{L}}
\renewcommand{\Pr}{{\mathbb{P}}}
\providecommand{\set}[1]{\left\{#1\right\}}
\providecommand{\uvec}{\mathbf{1}}
\providecommand{\Rang}{\mathrm{Rang}}
\providecommand{\Trace}{\mathrm{Trace}}
\providecommand{\Tr}{\mathrm{Tr}}
\providecommand{\CI}{\mathrm{CI}}
\providecommand{\asyvar}{\mathrm{AsyVar}}
\DeclareMathOperator{\Supp}{Supp}
\newcommand{\inputslide}[2]{{
    \usebackgroundtemplate{
     \includegraphics[page={#2},width=0.90\textwidth,keepaspectratio=true]
      %\includegraphics[page={#2},width=\paperwidth,keepaspectratio=true]
      {{#1}}}
    \frame[plain]{}
  }}
\newcommand\pperp{\perp\!\!\!\perp}
\newcommand\independent{\protect\mathpalette{\protect\independenT}{\perp}}
\def\independenT#1#2{\mathrel{\rlap{$#1#2$}\mkern2mu{#1#2}}}
\usepackage{bbm}
\providecommand{\Ind}{\mathbf{1}}
\newcommand{\sumjsi}{\underset{i<j}{{\sum}}}
\newcommand{\prodjsi}{\underset{i<j}{{\prod}}}
\newcommand{\sumisj}{\underset{j<i}{{\sum}}}
\newcommand{\prodisj}{\underset{j<i}{{\prod}}}
\newcommand{\sumobs}{\underset{i=1}{\overset{n}{\sum}}}
\newcommand{\sumi}{\underset{i=1}{\overset{n}{\sum}}}
\newcommand{\prodi}{\underset{i=1}{\overset{n}{\prod}}}
\newcommand{\prodobs}{\underset{i=1}{\overset{n}{\prod}}}
\newcommand{\simiid}{{\overset{i.i.d.}{\sim}}}
%\newcommand{\sumobs}{\sum_{i=1}^N}
%\newcommand{\prodobs}{\prod_{i=1}^N}
%\newcommand{\sumjsi}{\sum_{i<j}}
%\newcommand{\prodjsi}{\prod_{i<j}}
%\newcommand{\sumisj}{\sum_{j<i}}
%\newcommand{\prodisj}{\sum_{j<i}}

%\usepackage{appendixnumberbeamer}
\setbeamertemplate{footline}[frame number]
\setbeamertemplate{section in toc}[sections numbered]
\setbeamertemplate{subsection in toc}[subsections numbered]
\setbeamertemplate{subsubsection in toc}[subsubsections numbered]

%\makeatother
%\setbeamertemplate{footline}
%{
%    \leavevmode%
%    \hbox{%
%        \begin{beamercolorbox}[wd=.333333\paperwidth,ht=2.25ex,dp=1ex,center]{author in head/foot}%
%            \usebeamerfont{author in head/foot}\insertshortauthor
%        \end{beamercolorbox}%
%        \begin{beamercolorbox}[wd=.333333\paperwidth,ht=2.25ex,dp=1ex,center]{title in head/foot}%
%            \usebeamerfont{title in head/foot}\insertshorttitle
%        \end{beamercolorbox}%
%        \begin{beamercolorbox}[wd=.333333\paperwidth,ht=2.25ex,dp=1ex,right]{date in head/foot}%
%            \usebeamerfont{date in head/foot}\insertshortdate{}\hspace*{2em}
%            \insertframenumber{} / \inserttotalframenumber\hspace*{2ex} 
%        \end{beamercolorbox}}%
%       \vskip0pt%
 %   }
%   \makeatother
%\setbeamertemplate{navigation symbols}{}
\setbeamertemplate{itemize items}[ball]
%\setbeamertemplate{itemize items}{-}
%\newenvironment{wideitemize}{\itemize\addtolength{\itemsep}{10pt}}{\enditemize}
% \usepackage{eso-pic}
%\newcommand\AtPagemyUpperLeft[1]{\AtPageLowerLeft{%
%\put(\LenToUnit{0.9\paperwidth},\LenToUnit{0.9\paperheight}){#1}}}
%\AddToShipoutPictureFG{
%  \AtPagemyUpperLeft{{\includegraphics[width=1.1cm,keepaspectratio]{../logo-uga.png}}}
%}%
\def\figheight{3in}
\def\figwidth{4in}

%%Commands from Econometric Theory(Slides) by J. Stachurski.

\newcommand{\boldx}{ {\mathbf x} }
\newcommand{\boldu}{ {\mathbf u} }
\newcommand{\boldv}{ {\mathbf v} }
\newcommand{\boldw}{ {\mathbf w} }
\newcommand{\boldy}{ {\mathbf y} }
\newcommand{\boldb}{ {\mathbf b} }
\newcommand{\bolda}{ {\mathbf a} }
\newcommand{\boldc}{ {\mathbf c} }
\newcommand{\boldd}{ {\mathbf d} }
\newcommand{\boldi}{ {\mathbf i} }
\newcommand{\bolde}{ {\mathbf e} }
\newcommand{\boldp}{ {\mathbf p} }
\newcommand{\boldq}{ {\mathbf q} }
\newcommand{\bolds}{ {\mathbf s} }
\newcommand{\boldt}{ {\mathbf t} }
\newcommand{\boldz}{ {\mathbf z} }
\newcommand{\boldr}{ {\mathbf r} }

\newcommand{\boldzero}{ {\mathbf 0} }
\newcommand{\boldone}{ {\mathbf 1} }

\newcommand{\boldalpha}{ {\boldsymbol \alpha} }
\newcommand{\boldbeta}{ {\boldsymbol \beta} }
\newcommand{\boldgamma}{ {\boldsymbol \gamma} }
\newcommand{\boldtheta}{ {\boldsymbol \theta} }
\newcommand{\boldxi}{ {\boldsymbol \xi} }
\newcommand{\boldtau}{ {\boldsymbol \tau} }
\newcommand{\boldepsilon}{ {\boldsymbol \epsilon} }
\newcommand{\boldmu}{ {\boldsymbol \mu} }
\newcommand{\boldSigma}{ {\boldsymbol \Sigma} }
\newcommand{\boldOmega}{ {\boldsymbol \Omega} }
\newcommand{\boldPhi}{ {\boldsymbol \Phi} }
\newcommand{\boldLambda}{ {\boldsymbol \Lambda} }
\newcommand{\boldphi}{ {\boldsymbol \phi} }

\newcommand{\Sigmax}{ {\boldsymbol \Sigma_{\boldx}}}
\newcommand{\Sigmau}{ {\boldsymbol \Sigma_{\boldu}}}
\newcommand{\Sigmaxinv}{ {\boldsymbol \Sigma_{\boldx}^{-1}}}
\newcommand{\Sigmav}{ {\boldsymbol \Sigma_{\boldv \boldv}}}

\newcommand{\hboldx}{ \hat {\mathbf x} }
\newcommand{\hboldy}{ \hat {\mathbf y} }
\newcommand{\hboldb}{ \hat {\mathbf b} }
\newcommand{\hboldu}{ \hat {\mathbf u} }
\newcommand{\hboldtheta}{ \hat {\boldsymbol \theta} }
\newcommand{\hboldtau}{ \hat {\boldsymbol \tau} }
\newcommand{\hboldmu}{ \hat {\boldsymbol \mu} }
\newcommand{\hboldbeta}{ \hat {\boldsymbol \beta} }
\newcommand{\hboldgamma}{ \hat {\boldsymbol \gamma} }
\newcommand{\hboldSigma}{ \hat {\boldsymbol \Sigma} }

\newcommand{\boldA}{\mathbf A}
\newcommand{\boldB}{\mathbf B}
\newcommand{\boldC}{\mathbf C}
\newcommand{\boldD}{\mathbf D}
\newcommand{\boldI}{\mathbf I}
\newcommand{\boldL}{\mathbf L}
\newcommand{\boldM}{\mathbf M}
\newcommand{\boldP}{\mathbf P}
\newcommand{\boldQ}{\mathbf Q}
\newcommand{\boldR}{\mathbf R}
\newcommand{\boldX}{\mathbf X}
\newcommand{\boldU}{\mathbf U}
\newcommand{\boldV}{\mathbf V}
\newcommand{\boldW}{\mathbf W}
\newcommand{\boldY}{\mathbf Y}
\newcommand{\boldZ}{\mathbf Z}

\newcommand{\bSigmaX}{ {\boldsymbol \Sigma_{\hboldbeta}} }
\newcommand{\hbSigmaX}{ \mathbf{\hat \Sigma_{\hboldbeta}} }
\newcommand{\betahat}{\hat{\beta}}
\newcommand{\gammahat}{\hat{\gamma}}
\newcommand{\Uhat}{\hat{U}}
\newcommand{\Vhat}{\hat{V}}
\newcommand{\epsilonhat}{\hat{\epsilon}}
\newcommand{\sigmahat}{\hat{\sigma}}
\newcommand{\Sigmahat}{\hat{\Sigma}}
\newcommand{\Gammahat}{\hat{\Gamma}}

\newcommand{\RR}{\mathbbm R}
\newcommand{\CC}{\mathbbm C}
\newcommand{\NN}{\mathbbm N}
\newcommand{\PP}{\mathbbm P}
\newcommand{\EE}{\mathbbm E \nobreak\hspace{.1em}}
\newcommand{\EEP}{\mathbbm E_P \nobreak\hspace{.1em}}
\newcommand{\ZZ}{\mathbbm Z}
\newcommand{\QQ}{\mathbbm Q}


\newcommand{\XX}{\mathcal X}

\newcommand{\aA}{\mathcal A}
\newcommand{\fF}{\mathscr F}
\newcommand{\bB}{\mathscr B}
\newcommand{\iI}{\mathscr I}
\newcommand{\rR}{\mathscr R}
\newcommand{\dD}{\mathcal D}
\newcommand{\lL}{\mathcal L}
\newcommand{\llL}{\mathcal{H}_{\ell}}
\newcommand{\gG}{\mathcal G}
\newcommand{\hH}{\mathcal H}
\newcommand{\nN}{\textrm{\sc n}}
\newcommand{\lN}{\textrm{\sc ln}}
\newcommand{\pP}{\mathscr P}
\newcommand{\qQ}{\mathscr Q}
\newcommand{\xX}{\mathcal X}

\newcommand{\ddD}{\mathscr D}


%\newcommand{\R}{{\texttt R}}
\newcommand{\risk}{\mathcal R}
\newcommand{\Remp}{R_{{\rm emp}}}

\newcommand*\diff{\mathop{}\!\mathrm{d}}
\newcommand{\ess}{ \textrm{{\sc ess}} }
\newcommand{\tss}{ \textrm{{\sc tss}} }
\newcommand{\rss}{ \textrm{{\sc rss}} }
\newcommand{\rssr}{ \textrm{{\sc rssr}} }
\newcommand{\ussr}{ \textrm{{\sc ussr}} }
\newcommand{\zdata}{\mathbf{z}_{\mathcal D}}
\newcommand{\Pdata}{P_{\mathcal D}}
\newcommand{\Pdatatheta}{P^{\mathcal D}_{\theta}}
\newcommand{\Zdata}{Z_{\mathcal D}}


\newcommand{\e}[1]{\mathbbm{E}[{#1}]}
\newcommand{\p}[1]{\mathbbm{P}({#1})}
% definition
\BeforeBeginEnvironment{definition}{
  \setbeamerfont{block title}{series=\bfseries}
  \setbeamercolor{block title}{fg=MidnightBlue,bg=white}
  \setbeamercolor{block body}{fg=black, bg=gray!10}
}
\newtheorem*{definition*}{Definition}
\BeforeBeginEnvironment{definition*}{
  \setbeamerfont{block title}{series=\bfseries}
  \setbeamercolor{block title}{fg=MidnightBlue,bg=white}
  \setbeamercolor{block body}{fg=black, bg=gray!10}
}

% theorem
\BeforeBeginEnvironment{theorem}{
  \setbeamerfont{block body}{shape=\itshape}
  \setbeamerfont{block title}{series=\bfseries}
  \setbeamercolor{block title}{fg=MidnightBlue,bg=white}
  \setbeamercolor{block body}{fg=black, bg=gray!10}
}
\newtheorem*{theorem*}{Theorem}
\BeforeBeginEnvironment{theorem*}{
  \setbeamerfont{block body }{shape=\itshape}
  \setbeamerfont{block title}{series=\bfseries}
  \setbeamercolor{block title}{fg=MidnightBlue,bg=white}
  \setbeamercolor{block body}{fg=black, bg=gray!10}
}

% definition_fr
\theoremstyle{definition}
\newtheorem{definition_fr}{Définition}%[section]
\BeforeBeginEnvironment{definition_fr}{
  \setbeamerfont{block title}{series=\bfseries}
  \setbeamercolor{block title}{fg=MidnightBlue,bg=white}
  \setbeamercolor{block body}{fg=black, bg=gray!10}
}
\newtheorem*{definition_fr*}{Définition}
\BeforeBeginEnvironment{definition_fr*}{
  \setbeamerfont{block title}{series=\bfseries}
  \setbeamercolor{block title}{fg=MidnightBlue,bg=white}
  \setbeamercolor{block body}{fg=black, bg=gray!10}
}
% theorem_fr
\newtheorem{theorem_fr}{Théorème}%[section]
\BeforeBeginEnvironment{theorem_fr}{
  \setbeamerfont{block body}{shape=\itshape}
  \setbeamerfont{block title}{series=\bfseries, shape = \upshape}
  \setbeamercolor{block title}{fg=MidnightBlue,bg=white}
  \setbeamercolor{block body}{fg=black, bg=gray!10}
}
\newtheorem*{theorem_fr*}{Théorème}
\BeforeBeginEnvironment{theorem_fr*}{
  \setbeamerfont{block body}{shape=\itshape}
  \setbeamerfont{block title}{series=\bfseries, shape = \upshape}
  \setbeamercolor{block title}{fg=MidnightBlue,bg=white}
  \setbeamercolor{block body}{fg=black, bg=gray!10}
}

% remark_fr
\theoremstyle{remark}
\newtheorem{remark_fr}{Remarque}%[section]
\BeforeBeginEnvironment{remark_fr}{
  \setbeamerfont{block title}{series=\bfseries, shape=\itshape}
  \setbeamercolor{block title}{fg=MidnightBlue,bg=white}
  \setbeamercolor{block body}{fg=black, bg=gray!10}
}
\newtheorem*{remark_fr*}{Remarque}
\BeforeBeginEnvironment{remark_fr*}{
  \setbeamerfont{block title}{series=\bfseries, shape=\itshape}
  \setbeamercolor{block title}{fg=MidnightBlue,bg=white}
  \setbeamercolor{block body}{fg=black, bg=gray!10}
}








\usepackage{color}
\usepackage{tikz}
\usetikzlibrary{shapes.geometric, arrows}
\usepackage{enumerate}   
\usepackage{multirow}
%\setbeamersize{text margin left=1.5em,text margin right=1.5em} 
%\setbeamersize{text margin left=1.2cm,text margin right=1.2cm} 
\setbeamersize{text margin left=1.5em,text margin right=1.5em} 
%\usepackage{xr}
%\externaldocument{Econometrie1_UGA_P2e}
  \usepackage{eso-pic}
%\newcommand\AtPagemyUpperLeft[1]{\AtPageLowerLeft{%
%\put(\LenToUnit{0.9\paperwidth},\LenToUnit{0.85\paperheight}){#1}}}
%\AddToShipoutPictureFG{
 % \AtPagemyUpperLeft{{\includegraphics[width=1.1cm,keepaspectratio]{logoUGA2020.pdf}}}
%}%

%\setbeamercolor{title}{fg=black}
%\setbeamercolor{frametitle}{fg=black}
%\setbeamercolor{section in head/foot}{fg=black}
%\setbeamercolor{author in head/foot}{bg=Brown}
%\setbeamercolor{date in head/foot}{fg=Brown}
\setbeamertemplate{section page}
{
    \begin{centering}
    \begin{beamercolorbox}[sep=11pt,center]{part title}
    \usebeamerfont{section title}\thesection.~\insertsection\par
    \end{beamercolorbox}
    \end{centering}
}
%\titlegraphic{\includegraphics[width=1cm]{logoUGA2020.pdf}}
\title[Regression linéaire]{\textbf{ \'ECONOM\'ETRIE \\ (UGA, S2)}}
\subtitle{\textbf{CHAPITRE 8:\\ MODÈLES LINÉAIRES POUR DONNÉES DE PANEL }}
\date{\today}
\author{Michal W. Urdanivia\inst{*}}
\institute{\inst{*}UGA, Facult\'e d'\'Economie, GAEL, \\
e-mail:
 \href{
     mailto:michal.wong-urdanivia@univ-grenoble-alpes.fr}{michal.wong-urdanivia@univ-grenoble-alpes.fr}}

%\titlegraphic{\includegraphics[width=1cm]{logoUGA2020.pdf}
%}

\begin{document}

%%% TIKZ STUFF
\usetikzlibrary{positioning}
\usetikzlibrary{snakes}
\usetikzlibrary{calc}
\usetikzlibrary{arrows}
\usetikzlibrary{decorations.markings}
\usetikzlibrary{shapes.misc}
\usetikzlibrary{matrix,shapes,arrows,fit,tikzmark}
\usetikzlibrary{shapes}
\usetikzlibrary{shapes.geometric, arrows}
\tikzset{   
        every picture/.style={remember picture,baseline},
        every node/.style={anchor=base,align=center,outer sep=1.5pt},
        every path/.style={thick},
        }
\newcommand\marktopleft[1]{
    \tikz[overlay,remember picture] 
        \node (marker-#1-a) at (-.3em,.3em) {};%
}
\newcommand\markbottomright[2]{%
    \tikz[overlay,remember picture] 
        \node (marker-#1-b) at (0em,0em) {};%
}
\tikzstyle{every picture}+=[remember picture] 
\tikzstyle{mybox} =[draw=black, very thick, rectangle, inner sep=10pt, inner ysep=20pt]
\tikzstyle{fancytitle} =[draw=black,fill=red, text=white]
\tikzstyle{observed}=[draw,circle,fill=gray!50]

\begin{frame}
\titlepage
\end{frame}
\begin{frame}
 \tableofcontents
    \end{frame}
%\begin{frame}
%\frametitle{Contenu}
%\tableofcontents[pausesections, pausesubsections]
%\end{frame}

%\section{Qu'est-ce que l’économétrie ? A quoi (à qui) ça sert ?}
%\frame{\sectionpage}
%\begin{frame}
%  \tableofcontents  
%\end{frame}

\section{Introduction}
\frame{\sectionpage}
\begin{frame}[allowframebreaks]{Données de panel}
\begin{itemize}
\item Les données de panel sont issues de l'observation d'unités d'observation (e.g., ménages, entreprises, personnes, pays, \ldots)
 au cours de plusieurs périodes. 
 \item Traditionnellement les unités sont appelées \emph{individus} et nous adopterons cette pratique.
\item Ainsi si $\boldw$ est un vecteur de variables dont les observation correspondent à un panel, 
les observations/données seront:
\begin{align*}
    \left\{\boldw_{it} : i = 1, \ldots N; t\in\{1, \ldots, T\}\right\}
\end{align*}
où:
\begin{enumerate}[$\star$]
    \item $i$ est l'indice des individus,
    \item $t$ est l'indice des périodes,
    \item $N$ est le nombre d'individus,
    \item $T$ est le plus grand nombre de périodes d'observation dans le panel.
\end{enumerate}
\end{itemize}

\end{frame}
\begin{frame}[allowframebreaks]{Remarques}
\begin{remark_fr}
\begin{enumerate}[$\star$]
\item Quand tous les individus sont observés un même nombre de périodes, 
leur nombre de périodes d'observation  $T_i$ est donc le même et égal à $T$. Plus précisément on distingue:
\begin{enumerate}[-]
    \item les panels cylindrés qui correspondent $T_i = T$, pour tout $i=1, \ldots, N$,
    \item les panels non-cylindrés qui correspondent $T_i \neq T$ pour au moins un individu.
\end{enumerate}
\item Dans un panel non-cylindré les individus peuvent commencer à être observés à des périodes différentes:
\[
\begin{array}{l|llll}
    &\multicolumn{4}{c}{T}\\
    \hline
i=1&\cdot&\boldw_{11}&\boldw_{12}&\cdot\\
i=2&\boldw_{21}&\boldw_{22}&\boldw_{23}&\boldw_{24}\\
i=3&\cdot&\cdot&\boldw_{32}&\boldw_{34}\\
i=4&\cdot&\boldw_{42}&\boldw_{43}&\boldw_{44}
\end{array}
\]
\item Des extension à plus de 2 dimensions sont possibles, e.g., données employeur-employé où 
les individus sont observés au cours du temps et dans les différentes emplois occupés.
\item Il est possible d'ignorer la distinction entre panel cylindrés et non-cylindrés 
tant que les observations manquantes dans ce derniers le sont de façon aléatoire.
\end{enumerate}
\end{remark_fr}
\framebreak

\begin{remark_fr}
\begin{enumerate}[$\star$]
    \item Il est commun dans un cours d'économétrie sur les données de panel de (commencer par) 
    considérer les cas où $N$ est grand relativement à $T$. Les raisonnements et résultats asymptotiques sont 
    alors obtenus pour $N\rightarrow +\infty$. Cela concerne souvent des panel microéconomiques(e.g., 
    sur des ménages, entreprises, personnes).
    \item Pour ce qui concerne les modèles statiques que nous étudierons pour commencer, 
    tous les résultats qui seront présentés s'appliquent au cas inverse où $N$ est relativement petit 
    par rapport $T$ et où on suppose que $T\rightarrow +\infty$(il faut néanmoins interchanger $N$ et $T$ dans les formules).
\end{enumerate}
\end{remark_fr}

\framebreak

\begin{remark_fr}
\begin{enumerate}[$\star$]
    \item Les données de panel ne sont pas les fait d'avoir des observations sur plusieurs périodes, 
    c.à.d., des 
    \emph{données en coupe répétées}. Ce sont des données où \emph{les mêmes individus 
    sont suivis au cours de différentes périodes}.  
    \item Nous verrons que cette caractéristique offre la possibilité de contrôler 
    l'hétérogénéité entre les individus et notamment les problèmes d'endogénéité 
    qu'elle peut générer. 
    \item Par rapport au données en coupe et au séries temporelles, les données de panel 
    permettent de corriger les corrélation fallacieuses résultant de variables omises/inobservables.
     Pour cela:
    \begin{enumerate}[-]
        \item On distingue une hétérogénéité inobservée dans la dimension individuelle du panel, 
        et une hétérogénéité inobservée dans la dimension temporelle. 
        \item Supposer que l'hétérogénéité individuelle inobservée est constante 
        dans la dimension temporelle et que l'hétérogénéité temporelle 
        est constante dans la dimension individuelle permet de contrôler 
        les biais qu'elles peuvent générer.
        \item Quand on suppose $N$ grand relativement à $T$ et $N\rightarrow +\infty$ pour les raisonnements
        asymptotiques il est courant de ne se concentrer que sur l'hétérogénéité individuelle.
    \end{enumerate}
\end{enumerate}
\end{remark_fr}
\end{frame}

\begin{frame}[allowframebreaks]{Exemple: fonction de production agricole}
    \begin{itemize}
        \item Soit une fonction de production agricole Cobb-Douglas 'log-linéairisée":
        \begin{align}
            y_i &= a_{0, 0} + a_{1, 0}l_i + a_{2, 0}k_i + u_i,
            \label{eq1}
        \end{align}
        où pour une entreprise $i$: 
        \begin{enumerate}[$\star$]
        \item $y_i$ le (log) du niveau de production, 
        \item $l_i$ et $k_i$ sont les logs de deux facteurs de production,
         travail et capital, 
         \item $u_i$ capte l'ensemble des facteurs de production inobservées(qualité du sol, qualités/capacités
         inobservées du travail, \ldots). 
        \end{enumerate}
         \item Ces facteurs inobservées peuvent être corrélés avec les facteurs observées:
         \[\Cov[u_i; l_i] \neq 0, \ \Cov[u_i; k_i] \neq 0\]
         Par exemple $\Cov(u_i; l_i)> 0$ si $u_i$ représente une qualité du travail associée 
         à une plus grande productivité.
         \item On sait que dans ce cas(voir cours passés), l'estimateur des MCO des paramètres $a_{0, 0},,a_{1, 0}, a_{2, 0}$, 
         ne sera pas convergent.
         \framebreak
         \item Considérons la version suivante de l'équation précédente sur données de panel:
         \begin{align}
            y_{it} &= a_{0, 0} + a_{1, 0}l_{it} + a_{2, 0}k_{it} + \underbrace{u_{it}}_{=\alpha_i + \gamma_t + \varepsilon_{it}},
            \label{eq2}
        \end{align}
        où dans la décomposition de $u_{it}$: 
        \begin{enumerate}[$\star$]
        \item $\alpha_i$ les facteurs de production inobservés invariants dans la dimension temporelle,
        \item $\gamma_t$ des chocs temporels inobservés communs à toutes les entreprises, 
        \item $\varepsilon_{it}$ 
        est un choc temporel inobservée spécifique à la firme $i$. 
        \end{enumerate}
        \item Les différences premières entre $t$ et $t-1$ associées à l'équation de $y_{it}$ dans \eqref{eq2} sont:
        \begin{align}
            \Delta y_{it} &= a_{1, 0}\Delta l_{it} + a_{2, 0}\Delta k_{it} + \Delta \gamma_t +\Delta \varepsilon_{it}.
            \label{eq3}
        \end{align}
        \item Pour $N\rightarrow +\infty$ et $T$ fixe l'estimateur des MCO des paramètres 
        de \eqref{eq3} est convergent dès lors en particulier que les régresseurs 
        en différences premières sont exogènes par rapport à $\Delta \varepsilon_{it}$.
        \item Notons que même avec des régresseurs en différences premières  endogènes
         on peut utiliser les donner de panel pour obtenir des estimateurs de VIs(exploitant la 
         causalité au sens de Granger) des paramètres.
    \end{itemize}
\end{frame}

\begin{frame}
    [allowframebreaks]{Exemple: demande de travail dynamique}
    \begin{itemize}
    \item\cite{Sargent1978} considère l'équation suivante pour représenter la demande de travail 
    d'une firme représentative(equation d'Euler):
    \begin{align*}
        \Delta l_t &= a_{0, 0} +  a_{1, 0}\Delta l_{t-1} + a_{2, 0}w_t+ a_{3, 0}\frac{y_t}{l_t}  
        + \underbrace{u_t}_{= \gamma_t + \varepsilon_t},
    \end{align*}
    où:
    \begin{enumerate}[$\star$]
        \item  $\Delta l_t$, $w_t$, et $y_t$ sont respectivement la variation de l'emploi, le taux de salaire, 
        et le niveau de production, 
        \item $\varepsilon_t$ est un terme d'erreur orthogonal à l'information en $t$ ou avant $t$.
        \item $\gamma_t$ est une composante du profit marginal de la firme inobservé 
        du point de vue de l'analyste mais observé par la firme.
        \item Cette dernière composante dans le terme d'erreur $u_t$ peut être corrélée 
        aux régresseurs de sorte que l'estimateur des MCO des paramètres ne sera pas convergent.
        \item Supposons que l'équation précédente soit spécifiée pour des données de panel:
        \begin{align*}
            \Delta l_{it} &= a_{0, 0} +  a_{1, 0}\Delta l_{i, t-1} + a_{2, 0}w_{it}+ a_{3, 0}\frac{y_{it}}{l_{it}}  
            + \underbrace{u_{it}}_{= \alpha_i + \gamma_{t} + \varepsilon_{it}},
        \end{align*}
        où l'on suppose l'absence de corrélation sérielle pour $\varepsilon_{it}$.
        \item Les différence premières sur cette équation donnent:
        \begin{align*}
            \Delta^2 l_{it} &= a_{1, 0}\Delta^2 l_{i, t-1} + a_{2, 0}\Delta w_{it}+ a_{3, 0}\Delta \frac{y_{it}}{l_{it}} 
            + \Delta \gamma_t + \Delta \varepsilon_{it},
        \end{align*}
        que l'on peut estimer par VIs et/ou la MMG.
    \end{enumerate}

    \end{itemize}
\end{frame}
\begin{frame}[allowframebreaks]{Un modèle général pour données de panel}
\begin{itemize}
    \item Considérons le modèle:
    \begin{align}
        y_{it}&= f(\bold\boldx_{it}, u_{it}, \bolda_0)
        \label{eq4}
    \end{align}
    où:
    \begin{enumerate}[$\star$]
    \item $f(\cdot)$ est une fonction connue,
    \item $\bolda_0$ est un vecteur de paramètres inconnus,
    \item $\bold\boldx_{it}$ est un vecteur de variables explicatives observées; 
    \item $u_{it}$ représente les déterminants inobservés.
    \end{enumerate}
     \item  Suivant ce qui a été indiqué dans les exemples, une spécification 
     courante pour $u_{it}$ est:
     \begin{align}
        u_{it} &=\alpha_i + \gamma_t + \varepsilon_{it}.
        \label{eq5}
     \end{align}
\end{itemize}
\end{frame}

\begin{frame}[allowframebreaks]{Problème du paramètre incident}
\begin{itemize}
    \item Dans un modèle pour données de panel où les erreurs se décomposent comme dans \eqref{eq5}, 
    $\alpha_1, \ldots, \alpha_N$, et $\gamma_1, \ldots, \gamma_T$ peuvent être traités comme des 
    paramètres à estimer. 
    \item En ce sens il peuvent être vus comme mesurant les effets de $N-1$ 
    dummies individuelles et $T-1$ dummies temporelles. 
    \item Dans un cadre où l'on suppose que $N\rightarrow +\infty$ et que $T$ est fixe, 
    le nombre de paramètres $\alpha_1, \ldots, \alpha_N$ augmentera à la même 
    vitesse que $N$ ce qui empêche d'en obtenir des estimateurs convergents.
    \item Cela peut en outre compromettre l'estimation convergente de $\bolda_0$.
    
    \framebreak

    \item \textbf{Plus généralement}:
    \begin{enumerate}[$\star$]
    \item Étant donné un modèle caractérisé par des paramètres 
    $\boldtheta_0$ et un échantillon de $N$ observations, 
    \item on est en présence d'un problème de paramètre incident si:
    \[
        \text{quand} \ N \rightarrow +\infty \ \text{nous avons} \ \dim(\boldtheta_0) \rightarrow +\infty.
    \]
    \item \cite{NeymanScott1948} établissent qu'en règle général on ne peut avoir d'estimateur convergent 
    de $\boldtheta$.
    \item Supposons que $\boldtheta_0  = (\boldalpha, \bolda_0)$ où 
    $\boldalpha$ est le vecteur d'éléments $\alpha_1, \ldots, \alpha_N$, 
    et que $N\rightarrow+\infty$ alors bien que $\dim(\bolda_0)$ demeure fixe 
    $\dim(\boldalpha)\rightarrow+\infty$.
    \item La question qui se pose concerne la possibilité d'avoir des estimateurs convergents de 
    $\bolda_0$ bien que ceux de $\alpha$ ne puissent pas l'être.
    \end{enumerate}
\end{itemize}
\end{frame}
\begin{frame}[allowframebreaks]{Types de modèles pour données de panel}
\begin{itemize} 
    \item Il est  utile de distinguer des types différents de modèles pour données de panel car 
    les méthodes d'estimation présentent souvent des propriétés différentes selon le type de modèle considéré.  
    \item Cette distinction s'appuie sur deux critères:
    \begin{enumerate}
        \item additivité/non-additivité des erreurs,
        \item modèles statiques/dynamiques.
    \end{enumerate}
\end{itemize}
\end{frame}
\begin{frame}[allowframebreaks]{additivité/non-additivité des erreurs}
        \begin{itemize}
            \item Par exemple:
            \begin{align*}
                y_{it}&= h(\bold\boldx_{it}, \bolda_0) + u_{it} \ (\text{exemple d'erreurs additifs}),\\
                y_{it}&=\max\left\{ \bold\boldx_{it}\bolda_0 + u_{it}; 0\right\} \ (\text{exemple d'erreurs non-additifs}).
            \end{align*}
            \item Le problème du paramètre incident se présente de manière différentes
             selon que les erreurs sont additifs ou non.
             \item Un cas important sont les modèles non-linéaires statiques où 
             les erreurs sont non-additifs et où l'estimateur à effets 
             fixes qui est convergent dans le cas des modèles linéaires statiques ne l'est plus.
        \end{itemize}
    \end{frame}

        \begin{frame}[allowframebreaks]{Modèles statiques/non-statiques}
            \begin{itemize}
            \item Cette distinction dépends de la relation entre les variables $\{\bold\boldx_{it}\}$ et 
            les composantes $\{\varepsilon_{it}\}$ du modèle. 
            \item Dans un \textbf{modèle statique} $\bold\boldx_{it}$ est supposé \textbf{strictement exogène} par quoi 
            on entends qu'en $t$ il n'inclut pas des variables endogènes \emph{retardées}(c.à.d, correspondant aux périodes précédentes:
            \begin{align*}
                \Exp[u_{it}\boldx_{i, t+ s}]&=0 \ \text{pour tout} \ s\in\{\ldots, -2, -1, 0, 1, 2, \dots\}.
            \end{align*}
            \item Dans un \textbf{modèle dynamique} les variables endogènes retardées peuvent faire partie 
            des régresseurs. Par exemple $y_{i, t-1}$ peut faire partie de $\bold\boldx_{it}$. Comme alors 
            $\boldx_{i, t+1}$ inclut $y_{it}$,
            \begin{align*}
                \Exp[u_{it}\boldx_{i, t+s}]&\neq 0.
            \end{align*}
            Et des estimateurs convergents et efficaces pour des modèles statiques 
            ne sont même plus convergents dans le cas de modèles dynamique
\end{itemize}
\end{frame}

\section{Modèles Statiques}
\frame{\sectionpage}

\begin{frame}[allowframebreaks]{Modèle}
    \begin{itemize}
        \item Considérons l'équation suivante d'un modèle pour données de panel:

        \begin{align}
        y_{it}&=\boldx_{it}^\prime\bolda_0+ u_{it}, \ u_{it} = \alpha_i + \gamma_t + \varepsilon_{it}.
        \label{eq6}
        \end{align}
        où: 
        \begin{enumerate}[$\star$]
        \item $\boldx_{it}$ est vecteur $K \times 1$ de régresseurs, 
        auquel on associe un vecteur de paramètres $\bolda_0$,
        \item $\alpha_i$ représente l'hétérogénéité inobservée dans la dimension 
        individuelle, souvent appelé \emph{effet fixe individuel},
        \item $\gamma_t$ représente l'hétérogénéité inobservée dans la dimension 
        temporelle, souvent appelé \emph{effet fixe temporel},
        \item $\varepsilon_{it}$ est l'hétérogénéité inobservée 
        variable dans le temps et entre les individus.
        \end{enumerate}       
    \end{itemize}
\end{frame}

\begin{frame}[allowframebreaks]{Dummies temporelles}
    \begin{itemize}
        \item Nous allons considérer les cas où $T$ et relativement petit par rapport à $N$ avec 
        $N\rightarrow +\infty$ pour les analyses asymptotiques. 
        \item Ce faisant pour faciliter l'écriture du modèle, il nous pouvons adopter 
        une notation qui permettent d'inclure les paramètres $\{\gamma_t\}_{t=1}^T$ dans $\bolda_0$.
        \item Pour cela nous allons considérons un indicatrice temporelle $d_{it}^s$ pour  la période $s$ telle que:
        \begin{align*}
        d_{it}^s = 1 \quad \textrm{si} \quad t = s; \quad \textrm{et} \quad D_{it}^s = 0 \quad \textrm{si}\quad t \neq s
        \end{align*}
        alors,
\begin{align*}
\gamma_t  = \sum_{s=1}^Td_{it}^s\gamma_s = \left[d_{it}^1, d_{it}^2, \ldots, d_{it}^T\right]\boldgamma 
= \boldd_{it}^\prime \boldgamma
\end{align*}
où $\boldgamma$ est le vecteur $T\times 1$ défini par $\boldgamma \equiv (\gamma_1, \gamma_2, \ldots, \gamma_T)^\prime$.
\item On peut alors écrire \eqref{eq6}:
\begin{align*}
    y_{it}&=\begin{bmatrix}
\boldx_{it}&\boldd_{it}
    \end{bmatrix}^\prime
    \begin{bmatrix}
        \bolda_0\\
        \boldgamma 
    \end{bmatrix}
    + \alpha_i + \varepsilon_{it}
\end{align*}
qui est la version de \eqref{eq6} avec les effets temporels pris en compte par le bias d'un 
vecteur de régresseurs incluant des dummies temporelles.  
\item Pour ne pas alourdir les notations nous allons écrire cette équation simplement: 
\begin{align}
    y_{it}&=\boldx_{it}^\prime\bolda_0 + \alpha_i + \varepsilon_{it},
    \label{eq7}
\end{align}
pour laquelle il sera entendu que $\boldx_{it}$ inclue des dummies temporelles, et 
que $\bolda_0$ inclue $\boldgamma$.
\item Comme nous supposons que $T$ et petit relativement à $N$ avec $N\rightarrow +\infty$ 
le problème du paramètre incident provient seulement de $\alpha_1, \ldots, \alpha_N$.
\item Une démarche similaire peut être adopté dans le cas où $T$ 
est grand relativement à $N$ et $T\rightarrow +\infty$.

\framebreak 

\item Le  modèle peut être présenté comme un système de $T$ equations(une equation pour  $t=1, 2, \ldots, T$) 
avec $n$ observations pour chacune des équations du système.
\begin{align*}
y_{i1} &= \boldx_{i1}^\prime\beta + \alpha_i + \varepsilon_{it}\\
y_{i2} &= \boldx_{i2}^\prime\beta + \alpha_i + \varepsilon_{i2}\\
 \vdots &\\
 y_{iT} &= \boldx_{iT}^\prime\beta + \alpha_i + \varepsilon_{iT}    
\end{align*}
\item Sous forme vectorielle on peut écrire:
\begin{align*}
\boldy_i &= \boldX_i\beta + \boldone_T\alpha_i +\boldvepsilon_i\
\end{align*}
où
\begin{enumerate}[$\star$]
\item  $\boldy_i $ et $\boldvepsilon_i$   sont de vecteurs $T\times 1$, 
\item $\boldone_T$ est un vecteur  $T\times 1$ avec tous ses éléments égaux à 1; 
\item $\boldX_i$ est une matrice  $T\times K$. 
\end{enumerate}

\framebreak 

\item Une notation similaire peut être utilisée pour des panels non-cylindrés.
\begin{enumerate}[$\star$]
\item Soit  $w_{it} \in\{0, 1\}$; une indicatrice pour l'événement  "l'individu $i$ est observé à la période $t$". 
\item Alors:
\begin{align*}
\boldy_i &= \boldX_i \bolda_0 + \boldw_i \alpha_i + \boldvepsilon_i,
\end{align*}
où:
\begin{align*}
    \underset{T\times 1}{\boldy_i} &= \begin{bmatrix}
    w_{i1}y_{i1}\\
    w_{i2}y_{i2}\\
    \vdots\\
    w_{iT}y_{iT}
    \end{bmatrix}
    \quad , \quad
    \underset{T\times K}{\boldX_i} =  \begin{bmatrix}
    w_{i1}\boldx_{i1}^\prime\\
    w_{i2}\boldx_{i2}^\prime\\
    \vdots\\
    w_{iT}\boldx_{iT}^\prime
    \end{bmatrix}
    \quad , \quad
\underset{T\times 1}{\boldw_i} = \begin{bmatrix}
w_{i1}\\
w_{i2}\\
\vdots\\
w_{iT}
\end{bmatrix}
\end{align*}
\item Les expressions des estimators qui seront vus plus loin s'appliquent à cette définition
 de $\boldy_i$, $\boldX_i$, et $\boldw_i$.
\end{enumerate}
    \end{itemize}
\end{frame}    

\begin{frame}[allowframebreaks]{Exogénéité stricte(modèle statique)}
\begin{condition}(\textbf{\underline{Exogénéité stricte}})
\begin{align*}
    \Exp[\boldx_{it}\varepsilon_{is}] &= 0, \ \text{pour tout} \ (t,s)\in\{1, 2,\ldots, T\}.
\end{align*}
\end{condition}
\begin{itemize}
\item Cette condition ne sera pas vraisemblable sur des modèles incluant des régresseurs endogènes retardés:
 e.g.,$y_{i, t-1}$,$y_{i, t-2}$. 
 \item Par exemple, supposons que $y_{i,t-1}\subset \boldx_{it}$. 
 \item Le modèle énonce que $y_{i,t-1}$ depends de $\varepsilon_{i,t-1}$. 
 \item Il est alors clair que $\Exp[\boldx_{it} \varepsilon_{i,t-1}]\neq 0$.
\end{itemize}
\end{frame}

\begin{frame}[allowframebreaks]{Effets fixes ou effets aléatoires}
\begin{itemize}
    \item Un des problème les plus importants que les données de panel 
    permettent de traiter est celui de l'endogénéité due à l'hétérogénéité individuelle inobservée. 
    \item Dans le cadre de la décomposition de l'erreur du modèle dans \eqref{eq6} cela concerne 
    le problème posé par:
    \begin{align*}
        \Exp[\boldx_{it}\alpha_i]&\neq 0.
    \end{align*}
    \item Deux approches alternatives sont en général adoptées:
    \begin{enumerate}
        \item par les \textbf{effets fixes},
        \item par les \textbf{effets aléatoires}.
    \end{enumerate}
\end{itemize}
\end{frame}

\begin{frame}[allowframebreaks]{Effets fixes}
 \begin{itemize}
\item Dans cette approche aucune condition contraignant la relation entre $\alpha_i$ et $\{\boldx_{it}\}_{t=1}^T$
 n'est posée. Par exemple, on ne supposera pas que $ \Exp[\boldx_{it}\alpha_i] = 0 $.
 \item Dans la mesure où la loi conditionnelle de $\alpha_i$ sachant $\{\boldx_{it}\}_{t=1}^T$ 
 n'est pas contrainte, un \textbf{modèle à effets fixes} peut être vu comme non-paramétrique par rapport 
 à cette loi.
 \item Typiquement les modèles à effets fixes s'appuient sur une transformation du modèle qui élimine 
 les effets individuels où les rend redondants dans une vraisemblance conditionnelle.
\end{itemize}
\end{frame}
\begin{frame}[allowframebreaks]{Effets aléatoires}
    \begin{itemize}
   \item Cette approche se caractérise par les conditions qui contraignent la loi conditionnelle 
   de $\alpha_i$  sachant $\{\boldx_{it}\}_{t=1}^T$.
   \item La condition la plus  forte est l'indépendance de $\alpha_i$  entre $\{\boldx_{it}\}_{t=1}^T$ avec 
   $\alpha_i\sim i.i.d.(0, \sigma_\alpha^2)$. Il est courant de qualifier cette cas d'approche à \text{effets aléatoires}.
   \item Néanmoins d'autres approches à effets aléatoires existent. Par exemple dans \cite{CHAMBERLAIN19841247} 
   un modèle à \textbf{effets aléatoires corrélés} pose:
   \begin{align*}
    \alpha_i &= \lambda_0 + \boldx_{i1}^\prime \lambda_1 +\ldots +\boldx_{i1}^\prime \lambda_T + e_i,
   \end{align*}
   où $e_i$ est indépendant de $\{\boldx_{it}\}_{t=1}^T$, et $\{\lambda_t\}_{t=1}^T$ 
   sont des paramètres à estimer avec $\bolda_0$.
   \item C'est une \textbf{approche paramétrique} car elle dépends 
   sur une condition quant à la loi de $\{\boldx_{it}\}_{t=1}^T$ et $\alpha_i$.
\end{itemize}
\end{frame}

\begin{frame}[allowframebreaks]{Avantages relatifs entre les approches à EF et à EA}
    \begin{itemize}
        \item L'approche à EF est plus robuste qu'une approche à EA ou EA corrélés
         dans la mesure où ces dernières dépendent de la validité de la 
         condition imposé sur la relation entre effets individuels et régresseurs.
          Si cette condition n'est pas satisfaite 
          les estimateurs dans ces approches  ne seront probablement pas convergents.
          \item Les transformations utilisées dans les approches à EF peuvent éliminer 
          la variabilité dans l'échantillon des régresseurs exogènes. 
          Cela peut rendre les résultats moins précis que ceux d'une approche à EA(quand les conditions dans 
          celui-ci sont valides). 
          \item Pour certains modèles, il n'y pas de méthode d'estimation convergent par EF. 
          C'est par exemple le cas des modèles non-linéaires dynamiques. Voir par exemple \cite{Chamberlain2010}.
    \end{itemize}
\end{frame}

\section{Estimation sous exogénéité par rapport aux effets individuels}
\frame{\sectionpage}

\begin{frame}[allowframebreaks]{Estimation par MCO}
\begin{itemize}
    \item On commence avec le cas le plus simple où l'effet individuel 
    et les régresseurs ont une corrélation nulle:
    \begin{align*}
        \Exp[\boldx_{it}\alpha_i] &= 0.
    \end{align*}
    \item Dans ce cas, un estimateur convergent est simplement l'estimateur des MCO:
    \begin{align*}
    \hat{\bolda}^{MCO} &= \left[\sumiN\sumtTT \boldx_{it}\boldx_{it}^\prime\right]^{-1}\sumiN\sumtTT \boldx_{it}y_{it}\\
               &= \left[\sumiN \boldX_i^\prime \boldX_i\right]^{-1}\sumiN \boldX_i^\prime \boldy_i.
    \end{align*}
    \item Cet estimateur est souvent appelé \textbf{estimateur des MCO empilé} car nous empilons/traitons
     les observations comme s'il s'agissait de données en coupe.
     \item Cet estimateur n'est pas efficace en raison de la corrélation sérielle des erreurs $u_{it}$:
     \begin{align*}
        \Exp[u_{it}; u_{i, t-j}] &= \Exp\left[(\alpha_i + \varepsilon_{it})(\alpha_i + \varepsilon_{i, t-j})\right] = \Exp[\alpha_i^2] =\sigma_\alpha^2
    \end{align*}
    \item L'estimateur asymptotiquement efficace est l'estimateur de moindres carrés généralisés.
\end{itemize}
\end{frame}

\begin{frame}[allowframebreaks]{Estimateur des MCG}
    \begin{itemize}
        \item Pour présenter cet estimateur, écrivons d'abord le modèle dans la notation matricielle suivante:
        \begin{align*}
            \boldy &= \boldX \bolda_0 + \boldu
            \end{align*}
            où:
            \begin{align*}
            \underset{(NT\times 1)}{\boldy}  = \begin{bmatrix}
            \boldy_1\\
            \boldy_2\\
            \vdots\\
            \boldy_n
            \end{bmatrix}
            ,\quad
            \underset{(NT\times K)}{\boldX}  = \begin{bmatrix}
            \boldX_1\\
            \boldX_2\\
            \vdots\\
            \boldX_n
            \end{bmatrix}
            ,\quad
            \underset{(NT\times 1)}{\boldu}  = \begin{bmatrix}
            \boldu_1\\
            \boldu_2\\
            \vdots\\
            \boldu_n
        \end{bmatrix}
            \end{align*}
    \item Soit $\boldVr_{\boldu}$ la variance de  $\boldu$. 
    \item Il est courant de définir l'estimateur des MCG comme l'estimateur des MCO du modèle transformé/pondéré suivant:
    \begin{align*}
        \left[\boldVr_{\boldu}^{-1/2}\boldy \right]&= \left[\boldVr_{\boldu}^{-1/2}\boldX\right]\bolda_0 + \left[\boldVr_{\boldu}^{-1/2}\boldu\right]
    \end{align*}
    \item Nous obtenons des expressions de $\boldVr_{\boldu}$ , de $\boldVr_{\boldu}^{-1/2}$, 
    et des variables transformées
    $\left[\boldVr_{\boldu}^{-1/2}\boldy \right]$ et $\left[\boldVr_{\boldu}^{-1/2}\boldX \right]$. 
    \item Nous avons:
    \begin{align*}
    \boldVr_{\boldu} &= \Exp\left[\boldu \boldu^\prime\right] = \Exp
    \begin{bmatrix}
    \boldu_1\boldu_1^\prime &\boldu_1\boldu_2^\prime &\ldots &\boldu_1\boldu_N^\prime\\
    &\boldu_2\boldu_2^\prime&\ldots&\boldu_2\boldu_N^\prime\\
    &&\ddots&\vdots\\
    &&&\boldu_N\boldu_N^\prime
    \end{bmatrix}
    \end{align*}

    \item  Quand $\boldu_i$ est homoscédastique 
    sur la dimension individuelle et $\Exp\left[\boldu_i\boldu_j^prime\right] = 0$ pour $i \neq j$,
 
    \begin{align*}
    \boldVr_{\boldu} = 
    \begin{bmatrix}
    \boldOmega&\boldzero&\ldots &\boldzero\\
    \boldzero&\boldOmega&\ldots&\boldzero\\
    \vdots&\boldzero&\ddots&\vdots\\
    \boldzero&\vdots&\ldots&\boldOmega
    \end{bmatrix}
   = \Id_N\otimes\boldOmega
   \end{align*}
   où $\boldOmega$ est la matrice $T\times T$ $\Exp\left[\boldu_i\boldu_i^\prime\right]$.
\end{itemize}
\end{frame}

\begin{frame}[allowframebreaks]{QMCG quand les erreurs ne sont pas i.i.d.}
    \begin{itemize}
        \item On considère que $\boldSigma_{\boldu} = \Id_N\otimes\boldOmega$ où pour tout $i$ la matrice des variances-covariances 
        suivante ne fait pas l'objet de contraintes,
        \begin{align*}
        \boldOmega &= 
        \Exp
        \begin{bmatrix}
        u_{i1}^2&u_{i1} u_{i2}&\ldots &u_{i1}u_{iT}\\
        &u_{i2}^2&\ldots&u_{i2}u_{iT}\\
        &&\ddots&\vdots\\
        &&&u_{iT}^2
        \end{bmatrix}
        \end{align*}
        \item L'estimateur des MCG est équivalent à l'estimateur des MCO du modèle transformé suivant:
        \begin{align*}
        \left[\Id_n \otimes \Omega^{-1/2}\right]\boldy &= \left[\Id_N \otimes \Omega^{-1/2}\right]\boldX\bolda_0 + \boldu^{\star}
        \end{align*}
        \item L'estimateur des moindres carrés quasi généralisés(MCQG) est alors obtenu comme estimateur des MCO dans 
        ce modèle transformé où $ \Omega$ a été préalablement estimé par un estimateur convergent:
        \begin{enumerate}[$\star$]
        \item Soit $\hat{\boldu}_i= \left[\hat{u}_{i1},  \hat{u}_{i2}, \ldots,\hat{u}_{iT}\right]^\prime$ 
        le vecteur des résidus dans l'estimation par MCO du modèle non-transformé(MCO empilés) 
        pour l"individu $i$.  
        \item Un estimateur de $\boldOmega$ est:
        \begin{align*}
        \hat{\boldOmega} &= N^{-1} \sumiN \hat{\boldu}_i\hat{\boldu}_i^\prime.
        \end{align*}
        \item L'estimateur des MCQG utilise cet estimateur de $\boldOmega$.
    \end{enumerate}
    \end{itemize}
\end{frame}

\section{Estimation sans contrainte sur les effets individuels: "différences premières", "within", etc}
\frame{\sectionpage}
\begin{frame}[allowframebreaks]{Non-convergence des MCO}
    \begin{itemize}
        \item On n'impose pas que $\Exp[\boldx_{it}\alpha_i] = 0$.
        \item L'estimateur des MCO(ou sa version efficace avec les MCG) précédent n'est plus convergent 
        car:
        \begin{align*}
            \Exp\left[\boldx_{it}(\alpha_i + \epsilon_{it})\right]& \neq 0.
        \end{align*}
        \item Nous allons étudier les estimateurs suivants:
        \begin{enumerate}[$\star$]
            \item MCO du modèle en différences premières.
            \item Estimateur à effets fixes "within": variables en différence par rapport aux moyennes dans la dimension individuelle.
            \item MCO avec dummies individuelles(LSDV).
            \item Estimateur à effets aléatoires corrélés de Chamberlain.
            \end{enumerate}
    \end{itemize}
\end{frame}

\begin{frame}[allowframebreaks]{Modèle en différences premières}
    \begin{itemize}
        \item Comme cela a été indiqué plus haut les méthodes dites à effets fixes(c.à.d., 
        n'imposant pas de contrainte sur la relation entre $\alpha_i$ et $\{\boldx_{it}\}_{i=1}^N$) 
        s'appuient sur une transformation du modèle qui supprime $\alpha_i$. 
        \item Un premier estimateur fondé sur cette approche est l'estimateur du modèle en différences 
        premières contenu dans les exemples introductifs.
        \item  Le modèle transformé considéré est:
        \begin{align*}
        \Delta y_{it} &= \Delta \boldx_{it}^\prime \bolda_0 + \Delta \varepsilon_{it}
        \end{align*}
        où par $\Delta w_{it}$ on représente ici
        la différence première dans la dimension temporelle d'une variable $w_{it}$,
        
        c.à.d., $\Delta w_{it} = w_{it}-w_{i, t-1}$. Pour des vecteurs $\boldw_{it}$, $\Delta\boldw_{it}$ 
        est simplement l'opérateur de $\Delta$ appliqué à chaque élément du vecteur.

        \item L'exogénéité stricte de $\boldx_{it}$ implique que $ \Exp\left[ \Delta \boldx_{it}  \Delta \varepsilon_{it}\right] = 0$:
        \begin{align*}
        \Exp\left[\Delta \boldx_{it}  \Delta \varepsilon_{it}\right] =&=\underbrace{\Exp[\boldx_{it} \varepsilon_{it}]}_{=0} 
        + \underbrace{\Exp[\boldx_{it} \varepsilon_{i, t-1}]}_{=0} + \underbrace{\Exp[\boldx_{i, t-1} \varepsilon_{it}]}_{=0}  +  
        \underbrace{\Exp[\boldx_{i, t-1} \varepsilon_{i, t-1}]}_{=0}=0,
        \end{align*}
        et par conséquent l'estimateur des MCO du modèle en differences premières est convergent.
        \begin{align*}
        \hat{\bolda}^{FD}_N = 
        \left[\sumiN \underset{t=2}{\overset{T}{\sum}}\Delta \boldx_{it} \Delta \boldx_{it}^\prime\right]^{-1}
        \left[\sumiN \underset{t=2}{\overset{T}{\sum}}\Delta \boldx_{it} \Delta y_{it}\right] &\underset{N\to +\infty}{\limp} \bolda_0
        \end{align*}
        \item Cet estimateur n'est pas cependant efficace (sauf quand $\varepsilon_{it}$ est une marche aléatoire), 
        et on peut définir une estimateur des MCQG du modèle en différences premières.
    \end{itemize}
\end{frame}

\begin{frame}[allowframebreaks]{Estimateur du modèle en "within"}
\begin{itemize}
    \item L'estimateur en "within" du modèle est un estimateur à effets fixes sur la transformation suivante du modèle:
    \begin{align*}
        y_{it} - \bar{y}_i &= \left(\boldx_{it} - \bar{\boldx}_i\right)^\prime\bolda_0 + (\varepsilon_{it}-\bar{\varepsilon}_i)
        \end{align*}
        où $\bar{y}_i := T^{-1}\sumtTT y_{t}$, $\bar{\boldx}_i := T^{-1}\sumtTT \boldx_{it}$, $\bar{\varepsilon}_i := T^{-1}\sumtTT \varepsilon_{it}$.
        \item L'estimateur en within est simplement l'estimateur des MCO de ce modèle 
        en différence par rapport aux moyennes dans la dimension temporelle.
        \begin{align*}
        \hat{\bolda}^{WG}_N&=\left(\sumiN\sumtTT \left(\boldx_{it} - \bar{\boldx}_i\right) \left(\boldx_{it} - \bar{\boldx}_i\right)^\prime\right)^{-1}
        \sumiN\sumtTT \left(\boldx_{it} - \bar{\boldx}_i\right)\left(y_{it} - \bar{y}_i\right)
        \end{align*}
\end{itemize}
\end{frame}

\begin{frame}[allowframebreaks]{Convergence de l'estimateur en "within"}
\begin{itemize}
    \item Cet estimateur est convergent dès lors qu'il n'y a pas de colinéarité  de que $\left(\boldx_{it} - \bar{\boldx}_i\right) $ et
    $\Exp\left((\boldx_{it} - \bar{\boldx}_i) (\varepsilon_{it} - \bar{\varepsilon}_i)\right) = 0$.
    
    \item Notons que:
    \begin{align*}
    \Exp\left[(\boldx_{it} - \bar{\boldx}_i) (\varepsilon_{it} - \bar{\varepsilon}_i)\right]  &=\underbrace{\Exp[\boldx_{it}\varepsilon_{it}]}_{=0} +
    \underbrace{\Exp[\boldx_{it}\bar{\varepsilon}_i]}_{=0} + \underbrace{\Exp[\bar{\boldx}_i\varepsilon_{it}]}_{=0} + 
    \underbrace{\Exp[\bar{\boldx}_i\bar{\varepsilon}_i]}_{=0}\\
    &=0
    \end{align*}
    
    \item L'exogénéité stricte des régresseurs implique que toutes ces espérances sont zéro et 
    que par conséquent l'estimateur en within 
    est convergent.
    \item Cependant, l'estimateur de $\alpha_i$ ne l'est pas pour $T$ fixe. En effet:
    \begin{align*}
    \underset{N\to + \infty}{\plim} \hat{\alpha}_i &= T^{-1}\sumtTT\left(y_{it} - 
    \boldx_{it}^\prime\left(\underset{\plim}{N\to +\infty} \hat{\bolda}_N^{WG}\right)\right) 
    = T^{-1}\sumtTT\left(y_{it}- \boldx_{it}^\prime\beta\right)\\
    &=\alpha_i + \bar{\varepsilon}_i\neq \alpha_i
    \end{align*}
\end{itemize}
\end{frame}

\begin{frame}[allowframebreaks]{MCO avec dummies individuelle(LSDV)}
\begin{itemize}
\item On traite maintenant les $\{\alpha_i\}_{i=1}^N$ comme des paramètres à estimator $\bolda_0$ .

\item Pour cela écrivons le modèle comme suit:

\begin{align*}
\boldy &= \boldX\bolda_0 + \boldD \boldalpha  + \boldvepsilon =  
\begin{bmatrix}
\boldX&:&  \boldQ
\end{bmatrix}
\begin{bmatrix}
\bolda_0\\
\boldalpha
\end{bmatrix} + \boldvepsilon
\end{align*}
\begin{enumerate}[$\star$]
    \item $\boldQ$ est une matrice $N T\times N$ de dummies, une pour chaque individu,
    \item la ième colonne $\boldd$ contient les observations de la dummie pour l'individu $i$, 
    i.e., $1$ si l'observation appartient à $i$ et $0$ sinon
    \item  $\boldalpha$ est le vecteur associé aux 
    $\{\alpha_i\}_{i=1}^N$, i.e., $\boldalpha:=\left[\alpha_1, \ldots, \alpha_N\right]^\prime$.
\end{enumerate}
\item On appelle estimateur  LSDV( abbreviation pour "least squares dummie variables")
 l'estimateur des MCO de
$\begin{bmatrix}
\bolda_0\\
\boldalpha
\end{bmatrix}$, soit:

\begin{align*}
    \begin{bmatrix}
    \hat{\bolda}^{LSDV}_N\\
    \hat{\boldalpha}^{LSDV}_N
    \end{bmatrix}
    &=\left[
    \begin{bmatrix}
    \boldX&:&  \boldD
    \end{bmatrix}^\prime
    \begin{bmatrix}
    \boldX&:& \boldQ
    \end{bmatrix}\right]^{-1}
    \begin{bmatrix}
    \boldX&:& \boldQ
    \end{bmatrix}^\prime\boldy\\
    &=
    \begin{bmatrix}
    \boldX^\top \bold X & \boldX^\top\boldQ\\
    \boldQ^\top\boldX&\boldQ^\top\boldQ
    \end{bmatrix}^{-1}
    \begin{bmatrix}
    \boldX^\top\boldy\\
    \boldD^\top\boldy
    \end{bmatrix}
    \end{align*}
    
    \item Cet estimateur peut sembler exigeant du point 
    de vue calculatoire dans la mesure il repose sur
     l'inversion d'une matrice $(N +K)\times (N+ K)$ avec $N$ 
     pouvant devenir très grand sur des panels typiques.
    
     \item Cet estimateur peut être obtenu sans besoin d'inverser  "directement" cette matrice. 

\item On peut pour cela utiliser les propriétés des matrices with the particular structure of the matrices par
 blocs/partitionnées $\boldQ^\prime\boldQ$ et $\boldQ^\prime\boldX$. 


\item Appuyons nous sur le théorème  de Frish-Waugh-Lovell,  qui permet d'écrire l'estimateur des MCO comme suit:

\begin{align*}
\hat{\bolda}^{LSDV}_N
&=
\left[
\boldX^\prime\boldM_\boldQ\boldX
\right)^{-1}
\left(
\boldX^\prime\boldM_\boldQ\boldy
\right]
\end{align*}

où  $\boldM_\boldQ$ est une matrice idempotente:
\begin{align*}
\boldM_\boldQ &=\Id_{NT} - \boldQ\left[\boldQ^\prime\boldQ\right]^{-1}\boldQ^\prime\\
&= \left(\Id_n\otimes \Id_T\right) - \frac{1}{T}\left[\Id_n \otimes\boldone_T 
\boldone_T^\prime\right]\\
&= \Id_n\otimes\left[\Id_T -  \frac{1}{T} \boldone_T \boldone_T^\prime\right]
\end{align*}



\item When we pre-multiply $\boldy$ and $\boldX$ by $\boldM_\boldQ$, t
hese variables are transformed in deviations with respect to individual means: 
for $\boldM_\boldQ \boldy =: \boldy^{\star}$ and 
$\boldM_\boldQ \boldX =: \boldX^{\star}$, the $(i, t)$ element is given by:





When we pre-multiply $\boldy$ and $\boldX$ by $\boldM_\boldQ$, 
these variables are transformed in deviations with respect to individual 
means: for $\boldM_\boldQ \boldy =: \boldy^{\star}$ and $\boldM_\boldQ \boldX =: \boldX^{\star}$, 
the $(i, t)$ element is given by:
\begin{align*}
y^{\star}_{it} = y_{it}-\bar{y}_i, &\quad  x^{\star}_{it} = x_{it}-\bar{X}_i
\end{align*}

where $\bar{y}_i:= T^{-1}\sumt y_{it}$, $\bar{y}_i:= T^{-1}\sumt x_{it}$. 

\medskip

Since $\boldM_\boldQ$ is an idempotent matrix, we can write:
\begin{align*}
\betahat_{LSDV}  &=\left[ {\boldX^{\star}}^\prime\boldX^{\star}\right]^{-1}\left[{\boldX^{\star}}^\prime\boldy^{\star}\right]
\end{align*}
Therefore the LSDV estimator of $\bolda_0$ is equivalent to the OLS estimator of the transformed model:
\begin{align*}
y_{it}- \bar{y}_i &= (x_{it}- \bar{x}_i)^\prime\\bolda_0+ \varepsilon_{it}^{\star}
\end{align*}

The LSDV and the WG estimators are numerically equivalent (i.e., they are the same estimator). The WG approach is the 
efficient algorithm, from a computational point of view, of obtaining the LSDV.

\medskip

It is simple to show that the estimator of $\alpha_i$ is:

\begin{align*}
\hat\alpha}_{i, N}^{LSDV} &= T^{-1}\sumt(y_{it} - x_{it}^\prime\betahat)
\end{align*}






\end{itemize}
\end{frame}
\begin{frame}[allowframebreaks]{Références}
 \bibliographystyle{jpe}
  \bibliography{../Biblio}
   \end{frame}

    \end{document}
    