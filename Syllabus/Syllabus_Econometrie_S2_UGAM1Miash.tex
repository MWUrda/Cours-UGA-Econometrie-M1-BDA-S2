\input{../Preambule}

\newcommand{\bxi}{ \boldx_i }
\usepackage{tikz}
\usepackage{enumitem}
\usepackage{txfonts}
\usepackage{mathrsfs}
\usepackage{fancyhdr}



\usepackage{fancyhdr}
\usetikzlibrary{calc,decorations.pathmorphing,patterns}
\usepackage{listofitems}
\tikzstyle{mynode}=[thick,draw=blue,fill=blue!20,circle,minimum size=22]
\usepackage{amsaddr}
\usepackage{mathrsfs}

\makeatletter

\pgfdeclaredecoration{penciline}{initial}{
    \state{initial}[width=+\pgfdecoratedinputsegmentremainingdistance,auto corner on length=1mm,]{
        \pgfpathcurveto%
        {% From
            \pgfqpoint{\pgfdecoratedinputsegmentremainingdistance}
                            {\pgfdecorationsegmentamplitude}
        }
        {%  Control 1
        \pgfmathrand
        \pgfpointadd{\pgfqpoint{\pgfdecoratedinputsegmentremainingdistance}{0pt}}
                        {\pgfqpoint{-\pgfdecorationsegmentaspect\pgfdecoratedinputsegmentremainingdistance}%
                                        {\pgfmathresult\pgfdecorationsegmentamplitude}
                        }
        }
        {%TO 
        \pgfpointadd{\pgfpointdecoratedinputsegmentlast}{\pgfpoint{1pt}{1pt}}
        }
    }
    \state{final}{}
}
\makeatother

\pagestyle{fancy}
%\renewcommand{\subsection{mark}[1]{\markright{#1}{}}
\fancyhead{}
\fancyfoot{} 
%\fancyhead[LE,LO]{\tiny{\thepage}}
\fancyhead[C]{\small\textsc{UGA, M1 MIASH-BDA, S2}}
%fancyhead[CE,CO]{\tiny{\rightmark}}
\fancyhead[L]{\small\textsc{ÉCONOMÉTRIE 2}}
\fancyfoot[C]{\small{\thepage}}
%\fancyfoot[R]{\small \textcopyright \ \  \small\textsc
\fancyhead[R]{ \small\textsc{M. Urdanivia}}
%\renewcommand{\headrulewidth}{0pt}
\renewcommand{\footrulewidth}{0pt}

%\pagenumbering{roman}


\begin{document} 
\usetikzlibrary{positioning}
\usetikzlibrary{snakes}
\usetikzlibrary{calc}
\usetikzlibrary{arrows}
\usetikzlibrary{decorations.markings}
\usetikzlibrary{shapes.misc}
\usetikzlibrary{shapes}
%\tikzset{block/.style={draw, rectangle, fill=gray!20, 
%\tikzset{empty/.style={draw, rectangle, fill=none, tex
%\tikzset{line/.style={draw, -latex'}}
%\onehalfspace

%\includepdf{trame}

\begin{titlepage}
\centering
	%\includegraphics[width=0.15\textwidth]{logoUGA2020
    {\scshape\Large \textbf{\textsc{ÉCONOMÉTRIE 2}}\par}
	{\scshape\Large \textbf{\textsc{UGA, M1 MIASH-BDA, S2}}\par}
	\vspace{1cm}
	%{\scshape\large \textsc{Extremum Estimators(1)}\par}
	%\vspace{1cm}
	{\Large\bfseries \textsc{SYLLABUS} \par}
   % \vspace{1cm}   
	%{\Large\bfseries \textsc{RÉCAPITULATIF ET RÉSULTATS NON ASYMPTOTIQUES} \par}
    %{\Large\bfseries \textsc{EXAMEN(L3 MIASH, S2)} \par}
	%\vspace{1cm}
	{(\textsc{Cette version: \today})\par}
	\vspace{1cm}
	{\large \textsc{Michal Urdanivia}
	\footnote{Contact:  
	\href{mailto:michal.wong-urdanivia@univ-grenoble-alpes.fr}{michal.wong-urdanivia@univ-grenoble-alpes.fr}, 
	 Université de Grenoble Alpes,  Faculté d'\'Economie, GAEL.}\par}
	%\vfill
	%supervised by\par
	%Dr.~Mark \textsc{Brown}
%\vfill
% Bottom of the page
	
\end{titlepage}


\newpage

\tableofcontents

\newpage


\section*{Objectifs}
Ce cours poursuit l'enseignement d'économétrie du premier semestre.
 Il est organisé autour de deux parties\footnote{Le plan indiqué est indicatif et peut 
 être modifié au fur et à mesure de l'avancée du cours.}. 
 Dans la première partie le point de départ est le modèle linéaire du premier semestre et 
 on s'intéresse pour commencer au traitement du problème 
 d'endogénéité par la méthode des variables instrumentales en mettant l'accent sur 
 des extensions par rapport au premier semestre.  Enfin, on termine cette partie 
 par introduire à l'étude des méthodes pour données de panel
les plus couramment utilisées dans le cas de modèles linéaires.\\
Dans la deuxième partie on étudiera les modèles non-linéaires pour variables 
dépendantes limitées sur données en coupe.



 \section{Modèles linéaires: variables instrumentales, et méthodes pour données de panel}
 \subsection*{Résumé}
Notre point de départ ici est le problème posé pour l'inférence 
basée sur la méthode des moindres carrés ordinaires par la présence d'endogénéité. 
Le traitement de ce problème qui est au cœur de la microéconométrie moderne sera étudié 
dans le cadre conceptuel de la méthode des variables instrumentales(VIs). 
Celle-ci sera présentée et approfondie en traitant de l'hétéroscédasticité et des instruments faibles. 
Comme prolongement de l'étude du problème d'endogénéité traité par VIs nous étudierons les modèles à équations 
simultanées et les modèles linéaires pour données de panel. 
En termes méthodologiques, cette partie nous permettra d'étudier la méthode des moments généralisés.

\subsection*{Plan}

\begin{enumerate}[resume]
 \item Endogénéité et variables instrumentales: estimateur 
 des 2MC, tests d'Hausman et de Sargan,
instruments en présence d'hétéroscédasticité, instruments faibles.
\item Équations simultanées.
\item Modèles pour panels statiques: effets fixes et effets 
aléatoires, estimateurs de différences premières 
et estimateurs within. Inférence efficace en présence d'autocorrélation.
\item Panels dynamiques et exogénéité faible.
\end{enumerate}

\section{Modèles non linéaires pour variables dépendantes limitées sur données en coupe}
 
\subsection*{Résumé}
 Dans une première partie nous étudierons les modèles à variable dépendantes "limitées". Il s'agit de modèles 
 où les variables dépendantes ne peuvent pas être supposées continues étant par exemple discrètes(indicatrices de chômage, de remboursement de prêts, état de santé, choix de transport etc.),
 ou censurées(comme la consommation, qui prend nécessairement des valeurs positives mais est potentiellement nulle). 
 Les problèmes de sélection (offre de travail, sélection endogène d'échantillon) seront également abordés. 
 Dans le cadre de ces modèles non-linéaires, la méthode d'estimation utilisée est principalement 
 le maximum de vraisemblance.

 \subsection*{Plan}

\begin{enumerate}
\item Variables dépendantes discrètes.
\begin{enumerate}
\item Modèles logit et probit : identification, estimation, 
qualité du modèle, problème d'hétéroscédasticité et d'endogénéité.
\item Modèles polytomiques ordonnés et non ordonnés: 
logit multinomial, modèle logit conditionnel.
\end{enumerate}
\item Modèles de comptage: modèle de Poisson.
\item Modèles de censure et de sélection: modèles tobit simple, 
modèle de sélection généralisée, 
modèle de troncature.
\end{enumerate}


\section{Bibliographie}

Les notes écrites relatives au différents thèmes du cours seront mises à votre disposition sur le site/dépôt du cours(voir plus bas).
Elles ne suivent pas un ouvrage de référence mais vous pouvez les compléter avec  notamment:
\begin{itemize}[label = -]
\item \cite{Amemiya1985}.
\item \cite{ap2009}.
\item \cite{Hansen2017}.
\item \cite{Wooldridge2010}.
\end{itemize}
Des références supplémentaires seront parfois données dans le cadre des différents points traités.
En outre parmi les références vous trouverez aussi un certain nombre d'articles qui nous 
permettront d'illustrer certains points du cours. 


\section{Travail individuel}
Outre des exercices analytiques, il y aura des travaux d'application(empirique) des méthodes 
vues en cours(notamment dans les séances de TD/TP). Les correction se feront avec le langage 
Python\footnote{Notez que ce cours n'est pas un cours de Python. Vous aurez donc à vous former sur un certains nombre de points 
de façon autonome si vous souhaitez suivre les corrections.} mais vous pouvez utiliser 
le langage/logiciel de votre choix(R, Matlab, Julia, C++, \ldots). Ces applications concerneront souvent 
des articles de recherche et vous aurez à les lire et en faire des présentations et/où synthèses.

\section{Matériel du cours}

Les notes écrites, les codes Python, exercices, etc, se trouveront sur le dépôt Git suivant:

\url{https://github.com/MWUrda/Cours-UGA-Econometrie-M1Miash-S2}.

Moodle ne sera utilisé que pour vos retours de devoirs, et des "annonces" diverses sur le cours(e.g., dates de retour pour les devoirs, etc)
\bibliographystyle{jpe}
\bibliography{../Biblio.bib}
 \end{document}
