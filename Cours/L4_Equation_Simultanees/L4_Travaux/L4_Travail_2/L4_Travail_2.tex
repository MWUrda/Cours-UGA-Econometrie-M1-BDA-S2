\input{../../../../Preambule}

\newcommand{\bxi}{ \boldx_i }
\usepackage{tikz}
\usepackage{enumitem}
\usepackage{txfonts}
\usepackage{mathrsfs}
\usepackage{fancyhdr}



\usepackage{fancyhdr}
\usetikzlibrary{calc,decorations.pathmorphing,patterns}
\usepackage{listofitems}
\tikzstyle{mynode}=[thick,draw=blue,fill=blue!20,circle,minimum size=22]
\usepackage{amsaddr}
\usepackage{mathrsfs}

\makeatletter

\pgfdeclaredecoration{penciline}{initial}{
    \state{initial}[width=+\pgfdecoratedinputsegmentremainingdistance,auto corner on length=1mm,]{
        \pgfpathcurveto%
        {% From
            \pgfqpoint{\pgfdecoratedinputsegmentremainingdistance}
                            {\pgfdecorationsegmentamplitude}
        }
        {%  Control 1
        \pgfmathrand
        \pgfpointadd{\pgfqpoint{\pgfdecoratedinputsegmentremainingdistance}{0pt}}
                        {\pgfqpoint{-\pgfdecorationsegmentaspect\pgfdecoratedinputsegmentremainingdistance}%
                                        {\pgfmathresult\pgfdecorationsegmentamplitude}
                        }
        }
        {%TO 
        \pgfpointadd{\pgfpointdecoratedinputsegmentlast}{\pgfpoint{1pt}{1pt}}
        }
    }
    \state{final}{}
}
\makeatother

\pagestyle{fancy}
%\renewcommand{\subsection{mark}[1]{\markright{#1}{}}
\fancyhead{}
\fancyfoot{} 
%\fancyhead[LE,LO]{\tiny{\thepage}}
\fancyhead[C]{\small\textsc{UGA, M1 MIASH-BDA, S2}}
%fancyhead[CE,CO]{\tiny{\rightmark}}
\fancyhead[L]{\small\textsc{ÉCONOMÉTRIE 2}}
\fancyfoot[C]{\small{\thepage}}
%\fancyfoot[R]{\small \textcopyright \ \  \small\textsc
\fancyhead[R]{ \small\textsc{M. Urdanivia}}
%\renewcommand{\headrulewidth}{0pt}
\renewcommand{\footrulewidth}{0pt}

%\pagenumbering{roman}


\begin{document} 
\usetikzlibrary{positioning}
\usetikzlibrary{snakes}
\usetikzlibrary{calc}
\usetikzlibrary{arrows}
\usetikzlibrary{decorations.markings}
\usetikzlibrary{shapes.misc}
\usetikzlibrary{shapes}
%\tikzset{block/.style={draw, rectangle, fill=gray!20, 
%\tikzset{empty/.style={draw, rectangle, fill=none, tex
%\tikzset{line/.style={draw, -latex'}}
%\onehalfspace

%\includepdf{trame}

\begin{titlepage}
\centering
	%\includegraphics[width=0.15\textwidth]{logoUGA2020
    {\scshape\Large \textbf{\textsc{ÉCONOMÉTRIE 2}}\par}
	{\scshape\Large \textbf{\textsc{UGA, M1 MIASH-BDA, S2}}\par}
	\vspace{1cm}
	%{\scshape\large \textsc{Extremum Estimators(1)}\par}
	%\vspace{1cm}
	{\Large\bfseries \textsc{SYSTÈMES LINÉAIRES D'ÉQUATIONS SIMULTANÉES: TRAVAIL 2} \par}
   % \vspace{1cm}   
	%{\Large\bfseries \textsc{RÉCAPITULATIF ET RÉSULTATS NON ASYMPTOTIQUES} \par}
    %{\Large\bfseries \textsc{EXAMEN(L3 MIASH, S2)} \par}
	%\vspace{1cm}
	{(\textsc{Cette version: \today})\par}
	\vspace{1cm}
	{\large \textsc{Michal Urdanivia}
	\footnote{Contact:  
	\href{mailto:michal.wong-urdanivia@univ-grenoble-alpes.fr}{michal.wong-urdanivia@univ-grenoble-alpes.fr}, 
	 Université de Grenoble Alpes,  Faculté d'\'Economie, GAEL.}\par}
	%\vfill
	%supervised by\par
	%Dr.~Mark \textsc{Brown}
%\vfill
% Bottom of the page
	
\end{titlepage}


\newpage

\tableofcontents

\newpage

\section{Équations simultanées et 2MC }

Les questions suivantes sont issues de \cite{Wooldridge2010}(Problème 9.13, chapitre 9). 
Il s'agit d'une application sur des données et concernant une problématique développé par \cite{Romer1993QJE}. Les données peuvent être obtenues ici: 

\medskip

\url{http://qcpages.qc.cuny.edu/~rvesselinov/statadata/OPENNESS.DTA}

\medskip

L'objectif est d'étudier est d'étudier l'effet causal de l'ouverture d'une économie sur l'inflation. 
Le modèle considéré est le modèle à équations simultanées suivant:

\begin{align*}
	y_{(in,i)} &= a_0 + b_0 y_{(op, i)} + c_0 q_i + u_{(in,i)},\\
	y_{(op,i)} &= \alpha_0 + \beta_0 y_{(inf, i)} + \delta_0 q_i  + \gamma_0q_{(op, i)}+ u_{(op,i)},
\end{align*}
où pour un pays $i$: $y_{(in,i)}$ est le taux d'inflation("$inf$" dans le données), 
$y_{(op,i)}$ est une mesure de son degré d'ouverture("$open$" dans les données) , 
$q_i$ est son revenu par habitant en log("$lpinc$" dans les données), et $q_{(op, i)}$ 
est la superficie en log("$lland$" dans les données). Le revenu par habitant, et la superficie sont supposées exogènes.

\begin{enumerate}
\item À quelle condition la première équation est identifiée?
\item Proposez une stratégie d'estimation de la première équation par 2MC et faîtes la.
 Comparez le résultat de cette estimation à ceux obtenus par MC0 pour le paramètre $b_0$.
 \item On ajoute à la première équation le terme $d_0 y_{(op, i)}^2$. Expliquez 
 pourquoi on a besoin d'une VI supplémentaire. Proposez en une et testez sa significativité comme instrument.
 \item Estimez le modèle indiqué à la question précédente.
 \item Utilisez la méthode suivante pour estimer les paramètres de la première équation avec le terme supplémentaire de la question 3, (c.à.d., $a_0$, $b_0$, $c_0$, $d_0$): 
 \begin{enumerate}[label=(\roman*)]
 \item Estimez par MCO une équation linéaire avec comme variable dépendante $y_{(op,i)}$ et régresseurs un terme constant, $q_i$ et $q_{(op, i)}$ et calculez les valeurs ajustées/prédites $\hat{y}_{(op,i)}$.
 \item Estimez par MCO une équation linéaire avec comme variable dépendante $y_{(in,i)}$ et régresseurs un terme constant, $\hat{y}_{(op,i)}$, $\hat{y}_{(op,i)}^2$ et $q_i$. Comparez vos résultats à ceux de la question 3.
 \end{enumerate}
 
 
\end{enumerate}

\section{Équations simultanées et MMG}

On considère un modèle d'offre et demande s'inspirant du travail de \cite{AGI_RES_2000}. 
On utilise les mêmes données qui peuvent être obtenues ici:

\medskip

\url{https://people.brandeis.edu/~kgraddy/datasets/fishdata.dta}

\medskip

Le but est d'étudier la demande de poisson merlan sur le "Fulton Fish Market". 
Le modèle étudiée est le suivant:

\begin{align*}
	y_{d, i} & = a_0 p_i + \boldb_0^\prime z_{d, i} +  \boldd_0^\prime w_i + u_{d, i}\\
	y_{o, i} & = \alpha_0 p_i + \boldbeta_0^\prime z_{o, i} + \boldgamma_0^\prime w_i + u_{o, i},
\end{align*}

où la première équation est une équation de demande et la deuxième une équation d'offre avec $y_{d, i}$ la quantité demandé
et $y_{o, i}$ la quantité offerte, $p_i$ le prix $z_{d, i}$ et $z_{o, i}$, des 
vecteurs de régresseurs propres à chaque équation et $w_i$ des régresseurs communs aux deux équations.
On observe les quantités à l'équilibre telles que $y_{d, i} = y_{o, i} = y_i$ de sorte que le modèle
s'écrit:

\begin{align*}
	y_i & = a_0 p_i + \boldb_0^\prime z_{d, i} +  \boldd_0^\prime w_i + u_{d, i}\\
	y_i & = \alpha_0 p_i + \boldbeta_0^\prime z_{o, i} + \boldgamma_0^\prime w_i + u_{o, i}.
\end{align*}

Le but est donc d'estimer les élasticités prix de la demande et de l'offre, respectivement $a_0$ et $alpha_0$ avec des
 données sur la quantité(en log), et le prix(en log)(respectivement "$qty$" et "$price$" dans les données).
Le vecteur $z_{o, i}$ contient deux indicateurs des conditions météorologiques/océaniques("$stormy$", et "$mixed$" 
dans les données), le vecteur $z_{d, i}$ contient deux indicateurs des conditions météorologiques 
sur la côte("$rainy$", et "$cold$" dans les données) et des indicateurs des jours de marché($day1$, "$day2$", 
"$day3$", "$day4$"), les régresseurs communs aux deux équations $w_i$ sont seulement une constante.

\begin{enumerate}
\item Estimez par MCO les deux équations du système. 
\item Estimez le système dans une logique "équation par équation" par 2MC, 
expliquez la démarche(appuyez vous sur le cours).
\item Estimez le modèle par 3MC. 
\item Estimez le modèle par la MMG.
\end{enumerate}



\bibliographystyle{jpe}
\bibliography{../../../../Biblio.bib}
 \end{document}
