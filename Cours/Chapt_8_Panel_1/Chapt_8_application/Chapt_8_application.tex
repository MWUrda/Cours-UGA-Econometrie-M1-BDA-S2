\input{../../../Preambule}
\newcommand{\bxi}{ \boldx_i }
\usepackage{tikz}
\usepackage{enumitem}
\usepackage{txfonts}
\usepackage{mathrsfs}
\usepackage{fancyhdr}
\pagestyle{fancy}
%\renewcommand{\subsection{mark}[1]{\markright{#1}{}}
\fancyhead{}
\fancyfoot{} 
%\fancyhead[LE,LO]{\tiny{\thepage}}
\fancyhead[C]{\small\textsc{Économétrie}}
%fancyhead[CE,CO]{\tiny{\rightmark}}
\fancyhead[L]{\small\textsc{UGA}}
\fancyfoot[C]{\small{\thepage}}
%\fancyfoot[R]{\small \textcopyright \ \  \small\textsc
\fancyhead[R]{ \small\textsc{M. Urdanivia}}
%\renewcommand{\headrulewidth}{0pt}
\renewcommand{\footrulewidth}{0pt}

%\pagenumbering{roman}


\begin{document} 
\usetikzlibrary{positioning}
\usetikzlibrary{snakes}
\usetikzlibrary{calc}
\usetikzlibrary{arrows}
\usetikzlibrary{decorations.markings}
\usetikzlibrary{shapes.misc}
\usetikzlibrary{shapes}
%\tikzset{block/.style={draw, rectangle, fill=gray!20, 
%\tikzset{empty/.style={draw, rectangle, fill=none, tex
%\tikzset{line/.style={draw, -latex'}}
%\onehalfspace

\begin{titlepage}
\centering
	%\includegraphics[width=0.15\textwidth]{logoUGA2020
	{\scshape\Large \textsc{Économétrie: UGA}\par}
	\vspace{1cm}
	%{\scshape\large \textsc{Extremum Estimators(1)}\par}
	%\vspace{1cm}
	{\Large\bfseries \textsc{8: MODÈLES LINÉAIRES POUR DONNÉES DE PANEL} \par}
    \vspace{1cm}   
    {\Large\bfseries \textsc{Application} \par}
	\vspace{1cm}
	{(\textsc{Cette version: \today})\par}
	\vspace{1cm}
	{\large \textsc{Michal Urdanivia}
	\footnote{Contact:  
	\href{mailto:michal.wong-urdanivia@univ-grenoble-alpes.fr}{michal.wong-urdanivia@univ-grenoble-alpes.fr}, 
	 Université de Grenoble Alpes,  Faculté d'\'Economie, GAEL.}\par}
	%\vfill
	%supervised by\par
	%Dr.~Mark \textsc{Brown}
%\vfill
% Bottom of the page
	
\end{titlepage}


\newpage

\tableofcontents

\newpage

%\pagenumbering{arabic}
%\section{*{Objectifs du chapitre}

\section{Panel statiques}\footnote{Cette première partie reprend un exercice du cours d'Emmanuel Flachaire 
\href{https://sites.google.com/site/emmanuelflachaire/cours/econometrie-appliquee}{d'économétrie appliquée}
lui même issu d'un exercice dans Introduction to Econometrics de Stock et Watson.}

\subsection{Introduction}
Certains états des États-Unis ont acté des lois autorisant le port d'armes de ses citoyens,
mentalement compétents et sans casier judiciaire, appelées lois "shall-issue" ou "shall-carry".
Les lois sur l'émission du permis de port d'armes ont généré des débats sans fin. Ses partisans
considèrent que le port d'arme est une manière de dissuader les criminels, alors que ses
opposants affirment, au contraire, qu'il ne fait qu'accentuer la criminalité. Dans cet exercice,
nous tenterons d'analyser l'effet des lois sur l'émission du permis de port d'armes sur la
criminalité. Le fichier de données\footnote{Ces données ont été fournies par le professeur John Donohue de Stanford University et ont été utilisées dans 
\cite{AyresDonohueNBERw9336}.} est disponible sur le site de Mark Watson sous forme de fichier "Zip" avec les données et une description de celles-ci. Pour le récupérer allez ici: \\
\url{https://www.princeton.edu/~mwatson/Stock-Watson_4E/Guns.zip}\\
Il contient un panel cylindré de 50 États des États-Unis, plus le district de Columbia, pour la
période 1977-1999.

\subsection{Questions}
\begin{enumerate}
\item Estimez : 
\begin{enumerate}
\item la régression de $\mathrm{ln(vio)}$ sur $\mathrm{shall}$;\label{reg1}
\item la régression de $\mathrm{ln(vio)}$ ) sur $\mathrm{shall}$, $\mathrm{incarc\_rate}$, $\mathrm{density}$, $\mathrm{avginc}$, $\mathrm{pop}$, $\mathrm{pb1064}$, $\mathrm{pw1064}$ et $\mathrm{pm1029}$\label{reg2}.
\end{enumerate}
\item Interprétez le coefficient associé à $\mathrm{shall}$ dans la régression \ref{reg1} . Dans le contexte
réel, la valeur de ce paramètre estimé est-elle élevée ou faible?
\item L'ajout de variables de contrôle dans le modèle \ref{reg2} peut-il affecter l'effet de la loi
sur l'émission du permis de port d'armes ($\mathrm{shall}$) de la régression \ref{reg1} , en termes
de signification statistique ? Et en termes d'effet du paramètre estimé, dans le
contexte réel ?
\item Proposez une variable qui varie entre les États mais change probablement peu (ou
pas) avec le temps, et qui pourrait générer un biais d'omission dans la régression \ref{reg2} .
\item Écrire le modèle sous la forme d'un modèle pour données de panel avec seulement des effets individuels, 
et supposant que les effets fixes temporels font partie des régresseurs sous la forme de dummies.
\item Estimez le modèle sous une hypothèses d'effets aléatoires en rappelant le contenu de celle-ci(rappel: il s'agit 
en fait de l'estimateur des MCO dans les questions précédentes, mais vous pouvez en utiliser un plus efficace-voir cours-).
\item Estimez le modèle selon une hypothèse d'effets fixes en rappelant le contenu de celle-ci. L'estimateur peut être par exemple celui "within".
 Faîtes cela sans les dummies temporelles, et avec celles-ci. 
 \item Comparez vos différents résultats,
et précisez quels résultats sont selon vous les plus fiables, en argumentant(remarque: utilisez des écart-types 
estimés robustes).
\item Selon vous, quelles sont les principales menaces qui risquent de remettre en cause la
validité interne de cette analyse?
\item En vous basant sur votre analyse, quelles sont les conclusions que 
vous pourriez concevoir concernant l'effet de la loi sur l'émission du permis de port d'armes sur les taux
de criminalité ?
\end{enumerate}
\begin{theoreme}
jj
\end{theoreme}
\bibliographystyle{jpe}
\bibliography{../../../Biblio.bib}
 \end{document}
